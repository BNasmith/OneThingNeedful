\documentclass[12pt,a5paper,oneside]{book}
\usepackage{geometry}
\usepackage{palatino}
\setlength{\emergencystretch}{3em}

\usepackage[utf8]{inputenc}

% Next Step: Sort out the contents page and Roman numerals. 


\title{The Person of Jesus Christ}
\author{Hugh Ross Mackintosh}
\date{1912}

% Mackintosh, Hugh Ross. \textit{The Person of Jesus Christ}. London: Student Christian Movement, 1912.

\begin{document}

\frontmatter

\maketitle

\tableofcontents

\markboth{The Person of Jesus Christ}{The Person of Jesus Christ}

\section*{Publisher's Notice}

\begin{center}
\noindent\rule{4cm}{0.4pt}
\end{center}

\textsc{The} addresses here published in book form were
delivered originally at a summer conference of
the Student Christian Movement at Swanwick,
Derbyshire, in July, 1911. They were reproduced 
verbatim in \textsc{The Student Movement} for October,
November, and December of the same year, and
were felt by many people to be so valuable as to
make it advisable to print them in some more
permanent and accessible form. They have been
entirely re-written by the author for this purpose.

\chapter*{The Person of Jesus Christ}
\textsc{The} following pages are an attempt to
consider what is perhaps the most urgent 
religious question of our time---Who
was Jesus Christ, and what can be definitely
ascertained as to the purpose of His life? 
No subject could be more arresting; indeed,
for earnest observers of society, none could 
be more vividly engrossing at the present
hour. The world can hardly contain the 
books which are being written about Jesus. 
Not long since the present writer had occasion to read an article on Jesus, in a 
scholarly new Encyclop{\ae}dia of Religion, the 
author of which, coming finally to mention
the best literature, protested that out of the
\marginpar{9}
vast multitude of books he could name only
an insignificant and fragmentary selection.
Then two closely-printed columns were filled
with books of all sorts and sizes---biographies
of Jesus, controversial treatises, special studies,
books on the words of Jesus, on the character 
of Jesus, on His life, on His birth, on His death
and resurrection---most of them published 
within the last eight years. Thousands of
teachers teach about Jesus every day. Hundreds 
of preachers proclaim His Gospel. If a modern
theme could be named, that theme is Jesus 
Christ.

The subject-matter of this inquiry may 
be divided into three parts. First, we
shall endeavour to ascertain the most important 
facts known to be true regarding 
Jesus as He lived in Palestine. Next, we shall 
inquire as to the place and function filled by Him in Christian experience. Finally, there
remains the question---so vast and overwhelming---of
His relation to Almighty God. No 
problems more sublime could visit the human
mind; yet none are so intensely practical.
\marginpar{10}
For none could spring more directly out of
our personal attitude to the Gospel. We all 
of us know much regarding Jesus, and what
we know is our best possession; for no man
who has once absorbed a ray of Christ's light
can ever again become as though he had not 
heard His name. At the same time, we may
not as yet have focalized our impressions;
we may have delayed, so far, to gather what
we believe in one supreme, measured, and
coherent conviction, for which we can make a
stand, and which will satisfy the just demands
of intellectual consistency. It is to such an
inquiry, and to such an at least partial formulation 
of conclusions, that we are invited in the
present study.

At the very outset, however, it is necessary 
to repeat a familiar caution, or rather perhaps
to realize freshly the glory of a great promise.
Our insight into the fact of Jesus will depend 
essentially on our spiritual attitude and
temper. This is one chief matter, indeed,
which is covered by the word of Jesus,
''If any man willeth to do His will, he shall know
\marginpar{12}
of the teaching, whether it be of God'' (St. 
John vii. 17). Instinctively we recognize that
the significance of Christ is not equally clear to
every one, is not in fact at all times equally 
clear to ourselves. Nor ought we to suppose 
that our appreciation of Him is at all singular
in this respect. The principle holds everywhere.
Obedience is \textit{always} the organ of spiritual knowledge.
Is it not a familiar fact that our assurance 
of immortality wavers and flickers in 
secret concord with our habits? Does not our
sense of God wax and wane with our loyalty
to duty and our practice of secret prayer? So is 
it with the apprehension of Jesus Christ; we may 
make it feebly dim or grandly and inspiringly
and self-evidently clear by the attitude we take 
to His claim upon our lives. It is right that
this should be emphasized at the beginning. 
For it is useless to ignore the truth that the man
who has no wish to be good will tell you, 
if he is candid, that for him Jesus Christ has 
no value or attractiveness of any kind. Christ
means something great, something overwhelming, something divine and all-sufficient only
\marginpar{13}
for the man who is dissatisfied with himself;
who has aimed at righteousness, and now, to
his shame and grief, stands self-convicted of 
failure. You cannot see the beauty or the
sense of the glowing cathedral window from
without; to behold the splendour and the
miracle you must stoop and enter: and in like
manner Christ remains unintelligible and valueless 
to all save those who, under the constraint
of righteousness, have dared to pass with Him 
into the sanctuary of conscience. To know for
certain who Christ is, we must first have
gathered ourselves up in a genuine moral effort
and been brave enough to look straight and
clear at the facts of our own character and of
the moral universe.

\mainmatter

\chapter{The Jesus of History} \marginpar{14}
\markboth{The Person of Jesus Christ}{The Person of Jesus Christ}


\textsc{The} self-consciousness of Jesus---His thought of
Himself, that is, and of His redeeming mission to
the world---is not merely the greatest fact which
concerns Him; it is the greatest fact in all
history. It is from this point, therefore, that
we ought to start. It is the obviously right
point of departure, since it furnishes at the
very outset the foot-hold we require in the
known actualities of the past. Not to build
up an edifice of speculation is our aim, for that
could only have the value and credibility of
fallible human logic; but rather to account,
reasonably and worthily, for the astounding
circumstance that this Man holds, and has
always held, the central place in the supreme
religion of the world. To this starting-point
there is just one possible objection, based on
the hypothesis that Jesus lived under a sheer
\marginpar{15}
delusion about Himself because He had taken
up certain grandiose but pathetically absurd
notions current in His own day and thus
came to regard His own Person as the fulfilment 
of fantastic Jewish anticipations of
a Saviour from heaven, while the disciples fell
so completely under His influence that, like the
followers of the Mahdi, they came to share the
hallucination. But this theory need not be considered 
seriously. In itself it is manifestly no
more than a wild guess, utterly out of keeping
with Jesus' acknowledged sanity and insight.
Not only so; to admit that it raises a real problem 
is equivalent to renouncing the attitude
even of moral reverence for Jesus. We could
no longer venerate One Whose life was built
round a pure mistake.

It is right to emphasize at the outset the
immense significance of the fact that Jesus
Christ should have had an absorbing consciousness 
of Himself, or rather of God and Himself
as bound up together. No man in his senses
would dream of employing the phrase ``God and
I,'' yet just this is Jesus' tone. He cannot
\marginpar{16}
think of Himself without thinking also of God
Who sent Him and Who is perpetually with
Him. Still more amazing, He cannot think of
God but that His mind instantly settles on
Himself as God's indispensable organ and
representative. ``My Father worketh hitherto,
and I work'' (St. John v. 17). Here comes the
strange note of ``God and I,'' which we should
feel it impossible to adopt, or even to imitate.
And it is not merely that His tone and
attitude is in these ways so different from ours;
it is wholly unlike anything to be found else
where in religious history. Take Buddha.
When Buddha dies, he gives instructions that
his disciples may forget him if only they remember 
his teaching and the way that he has shown
them. Or, again, take Socrates. What he is
concerned about at the end is the truth he has
given his life to elucidate. These two, it has
been rightly observed, are nearer to Jesus in
moral power and originality than is any other;
Plato speaks of Socrates, in the closing lines of
the \textit{Phaedo}, as the wisest and justest and best
of all men he has known; yet it is clear that it
\marginpar{17}
had not occurred to them to take a central
position in the affections and thoughts of man
kind. How different is it with Jesus! He
came to lead men to \textit{God}; and yet, as
Herrmann has expressed it, ``He knows no
more sacred task than to point them to His
own Person.'' Such was His confidence in His
power to redeem, whether from sin or death,
that He felt at liberty to thrust Himself thus
deliberately on the world's attention. The
Gospel could be uttered only in this way. The
good news for a world of impotence and misery
could not be proclaimed save by fixing men's
eyes upon Himself. When therefore we repair
first of all to the self-revelation of Jesus, we are
standing beside the very fountain-head of all
Christian religion.

The content of Jesus' self-consciousness is
of course infinitely profound and comprehensive,
but for our present purpose it may be most
easily considered under two main aspects or
divisions. In the first place, He definitely took
the role of Messiah; in the second place, He
claimed to be the Son of God.
\marginpar{18}

In regard to the first point---the Messiahship
of Jesus---one can almost hear the instinctive expostulation 
of the modern man who is conscious
of living in the twentieth century. ``To me,'' he
will say, ``the word `Messiah' means nothing.
It is an old Jewish word. Once, no doubt, it was
filled with the life-blood of a great patriotism,
but it is bleached white now.'' Most of us probably 
will understand this objection to the word,
and may indeed sympathize with it. Yet
after all we are here dealing not with words
but with things; and what must therefore be
pointed out is that whereas the mere word is old
and empty, the thing is eternal. For Jesus and
His countrymen, no single word had such a
burning intensity of meaning as ``Messiah.''
In claiming to be Messiah, indeed, He simply
used a Jewish term expressive of His place as
Saviour of the world. For what did Messiahship
imply? One thing is quite certain: it
did not imply anything easy, obvious, or
commonplace. So far from that, its significance
is final, awful, revolutionary. To put it briefly:
in Jesus' mind, as in the mind of every pious
\marginpar{19}
Jew, the Messiah was the Person in Whom all
the purposes of God were gathered up and
consummated. The last foundations of being
were in Him. All creation in heaven and on
earth, all the Divine ways of history, all time
and all eternity---they meet and converge in
this one transcendent Figure. Whoever turned
out to be Messiah would thereby be constituted
the hinge and pivot of the universe, the Person
on Whom everything turned in the relation of
God to man. Do you imagine these are claims
to be lightly raised? Are they claims, more
over, which one like Jesus would make at
random? And yet, knowing this to be the
meaning of the name, Jesus stood up and applied
it directly to Himself. I am He, He said; I
am the Sent of God, in Whom every promise
is answered and every human prayer fulfilled.

At once we can see how tremendously it
matters whether this transcendent belief about
Himself was or was not true. Suppose it true;
then we must come to a personal understanding
with Jesus regarding His significance for our
own lives. It sets us right in view of the last
\marginpar{20}
and highest moral responsibility. It compels us
to choose, to be for or against, to face toward
God in Christ or away from Him. Suppose it
false and mistaken, what then? Can we continue 
to reverence the man who was deceived?
Remember, this claim of Jesus has gone through
history like a sword, dividing households cruelly,
producing martyrdom and self-sacrifice on a
scale never before seen, drawing passionate faith
and love and hope from a million hearts in
every generation; well, then, if it is all a
mistake---if some one has blundered, and that
some one Jesus---can this leave us admiring
Him any longer? I will go further: must it
not even shake our trust in God? For consider 
the question yet once more in its sharpest
form. Here is the most influential Figure in
history, whose influence bids fair to endure as
long as the world itself; and we can see that for
His own mind the Messianic thought, with its
boundless implications, is vital and decisive
Now confront this with the supposed fact that
the belief is only a rather discreditable piece of
fanaticism. What light is flung thereby on the
\marginpar{21}
Divine government of the world? What sort
of universe is it in which such things can be,
in which the best and bravest and highest flows
thus from a mere hallucination? Surely in view
of such issues it is not too much to say that
\textit{everything} in Christian religion hangs on the
spiritual veracity of Jesus' profession of Messiahship.
The question is not peripheral; it is
central and supreme.

To proceed then: it was this controlling
consciousness of being \textit{sent}, sent by God as
absolute Deliverer, which interprets all the
greatest facts in our Lord's career. That career
was no irresponsible adventure; behind each
word, act, or movement lay the vast background
of a Messianic commission to mankind. In the
first place, this explains \textit{His amazing tone of}
\textit{moral authority}. If familiarity had not dulled
our feelings as we peruse the Gospels, we should
be unable to restrain our astonishment at the
sense of unprecedented and inimitable authority
which is manifest in Jesus. Obviously He
had no scruple in asking men for unreserved
loyalty to Himself. He tells them that they
\marginpar{22}
are to give up everything, to give it up at once,
to rise without a word and follow Him. Nothing
must be allowed to interfere with this, not
even the dearest ties of natural affection.
''If any man cometh unto Me and hateth not
his own father and mother, yea, and his own
life also, he cannot be My disciple'' (Luke xiv.
26). Not that it was a self-regarding claim.
Jesus calls no man that He may use him as a
tool. He has evidently no will of His own
except to do what is requisite for His appointed
mission; but the arresting fact is this, that He,
the Meek and Lowly of heart, should conceive
that mission as so bound up with His own
Person as to be unrealizable apart from Him.
To refuse Him is to forfeit eternal life; He has
therefore no option but to insist on submission
and obedience. We must take up our cross
behind Him: there is no other way. How
vivid, solemn, and transfixing are the words!
How they force us, if we have any seriousness
of purpose, to scrutinize anew this Person
Who tells us that to live rightly is to accept
His yoke; tells us so, indeed, as if nothing
\marginpar{23}
else were conceivable. Could He dare to press
our conscience so hard, if He were merely
one of ourselves, a pious, good man like a
thousand others? There are those who skate
across this problem easily, one had almost
said with levity. Yet it insists on being met
and solved. If Jesus was but one more human
unit, however noteworthy, is this tone of
ethical supremacy justified? Is it even tolerable? 
Is it not, rather, an outrage alike on
conscience and on truth? Here, then, we
realize once more, at a crucial point, how vast
are the moral interests bound up with the self-consciousness
of Christ. No man can read the
Gospels without becoming sensible that according 
to our Lord's own conviction He was
bringing those to whom He spoke into the
presence of the final moral obligation; with
the consequence that their attitude to Him
could be no question of taste, or accident, or
degree; it was a question, rather, of life and
death. We have only to face frankly the moral
authority of Christ, alone with conscience and
in a still hour, to have the staggering conviction
\marginpar{24}
thrust upon us that in truth this Man has a right
to the Name which is above every name.

The next fact made luminous by Jesus'
consciousness of Himself is \textit{His forgiveness of}
\textit{sin.} When we read that marvellous episode
in Mark ii., the healing of the paralytic---one
of the most significant passages in all the
Gospels---we are at once struck by the
fact that Jesus does not proclaim forgiveness
merely (which any Christians may do, and all
Christians ought to); \textit{He professes to impart it}.
He puts forgiveness, you may say, right into
the sufferer's heart; and when challenged as
to His prerogative, He replies by a miracle of
healing. Doubtless it may be said that in a
real sense we also forgive sin. In St. John's
Gospel we have the risen Lord's promise
or declaration: ``Whose soever sins ye forgive,
they are forgiven unto them; whose soever
sins ye retain, they are retained'' (xx. 23).
But observe the difference. When we proclaim 
pardon to sinful men---whether from the
pulpit or in fireside talk---we do it in view of
Jesus, the guarantee of Divine grace to all
\marginpar{25}
the guilty; when Jesus offers pardon in the
Gospels, it is in virtue of Himself. Not as
though He expected men to believe it apart from
what they knew of \textit{Him}. As it has been put:
''Jesus did not write the story of the Prodigal
Son on a sheet of paper for those who knew
nothing of Himself. He told it to men who
saw Him, and who, through all that He was,
were assured of the Father in heaven, of
Whom He was speaking.'' These guilty men
found pardon realized in Jesus; as He stood
before them, He was surety to their souls of
the forgiving love of God. The woman that
was a sinner (Luke vii.) became conscious in
His presence that He was the door of entrance
to a purer and better life; in Him, she felt,
the Father said to her aching heart, ``I am
thy salvation.'' If till then God had been a
name of fear, and past and present were disquieting, 
there now came to her, mediated by
Jesus' tone and look, the blest sense of a Divine
love mightier than her sin---that initial assurance
of forgiveness which makes all things new. A
voice said, ``Rise up and be God's child'';
\marginpar{26}
and in that swift realization of patient love
which yet would make no terms with sin, the
peace of reconciliation flowed down into her
soul. Thus Jesus pardoned sin.

Now it is only long familiarity which hides
from us the astounding character of forgiveness.
Nothing in the world is so purely supernatural;
in comparison the raising of the dead may be
called a trifle. Do we not feel how impossible
it is to forgive ourselves if anything real has
to be forgiven? Doubtless we can make an
apology to our own better nature, thus wiping
the offence, whatever it is, off the slate; but
mere honesty will confess that really this is
not forgiveness in the least. Strictly speaking,
we can no more forgive ourselves than we can
shake hands with ourselves or look into our own
eyes. Further, we are quite well aware that
although we may forgive a man the injury he
has done us, we can never forgive his sin. That
he must settle with Almighty God, and he
knows it. But the amazing fact is that
Jesus said to men: You can settle it with
Me. You can tell Me of your penitence,
\marginpar{27}
and I am able to grant the Divine forgiveness.
Not only did He say this; over and over again
He made good His words. In unnumbered
cases He lifted the burden from the bad conscience, 
took off the paralyzing touch of guilt,
and once for all flung wide the gate of righteousness 
to those who had bolted and barred it in
their own face. He claimed to open the prison-door
to the captives of despair; and by a word,
a look, a touch of holy love, He opened it, so
that in the power of His presence men stood
up, shook off their chains, and passed out \textit{free}.
We need not now pause to analyze the various
implications of such an act. But anyone can see
that Jesus could not have offered pardon to men
in His own Person---on His own account and
guarantee, as it were---if He had Himself been
conscious of sin; while on the other hand it is
sheerly unthinkable that one such as He could
have been sinful without knowing it. In the
Roman service of the Mass there comes a point
at which the celebrating priest, even in that
awful hour, makes confession of sin to the
congregation, begging them to pray God for him;
\marginpar{28}
but there is no such consciousness in Jesus.
He is aware that He needs no cleansing. Even
in the article of death He knows it. There is
no consciousness of sin; there is no memory of
sin; there is no fear of sin as a future contingency 
flowing from the weakness or short
coming of even the most distant past. Sinlessly
one with God, all His life He moved
among men, uttering the word of pardon to
the guilty, and uttering it with Divine effect.

The third point which in a real sense is
made intelligible by the Messianic consciousness
of Jesus is \textit{His working of miracles}. Every now
and then, as we learn from history, the Church
passes through a certain period when she is
more or less ashamed of the miracles of Jesus;
and beyond all question this is due in part to
inherited misconceptions of miracle as such.
It is thought to be a violation of law, a breach
of causation, or the like, and not unnaturally
definitions of this kind create a violent and
unfavourable prejudice. But it is coming to
be quite clearly understood that miracle need
not imply any violation of law, and that belief
\marginpar{29}
in miracle is simply another name for belief
in the Living God. To quote one of the most
acute of British philosophers, Professor A. E.
Taylor, ``There is no philosophical justification
for relegating the providential action of God
to the infinitely remote past, and refusing to
admit the possibility of incessant new departures. 
Nor have we any ground to declare
that the actual course of events is conformable
to `immutable laws.' This has an important
bearing on the reality of those unusual sequences
commonly called `miracles.' There is really
no reason why the most unusual things should
not be happening somewhere or other every
day. In fact, the wonder would be, not that there
should be `miracles,' but that there should be so
few of them.'' If we have rejected the impossible 
conception of the universe as a mechanical
system in which everything---including history
and all human action---is absolutely and fatally
determined, and if in addition to this we
believe in the Living God of Jesus, we are
quite at liberty to hold that miracles are both
possible and real.

\marginpar{30}
To return, however: my point is that whether
the Church is or is not ashamed of miracles,
it is at least obvious that Christ was not. On
the contrary, we may affirm with all reverence
that, coming forward as He did in the character
of Messiah, He would have been ashamed \textit{not}
to do wondrous works. Remember once again
what Messiahship implies and must imply.
The Messiah came to establish the Divine rule,
the Kingdom of God; to establish it in a world
not of sin merely, but of need, of pain, of
death, of despair. If there is one point upon
which scholars are agreed to-day, it is that
the Kingdom as Jesus conceived it was a new,
heavenly, supernatural order of redemption,
differing \textit{toto c{\oe}lo} from the old disappointing
order---an order of so transcendent a character
that sin and grief should be abolished within
its range, and the omnipotent love of God have
free play. Jesus knew that all this had been
eagerly expected by the best souls in each
generation, and, when He stood up to preach at
Nazareth, His first word was the announcement
that the expectation was now fulfilled. The
\marginpar{31}
saying of the French monarch is familiar:
\textit{L'etat, c'est moi}---Myself am the State. Take
away the arrogance and falsehood, and we have
precisely the message of Jesus to those who
heard Him: I Myself am the Kingdom. God's
reign is begun in My presence among you.
What the miracles of Jesus meant therefore to
His own mind was simply that the first dawning
gleams of the new day had begun to shine.
The vast novel powers of the new order,
and its revolutionizing energies, were now firmly
planted in the world in His Person; and as
His ministry broadened out from more to more,
He was conscious of His power to work what
has been called ``the comprehensive miracle
of redemption,'' not only forgiving all our
iniquities but healing all our diseases.

And now we come to an idea of incalculable
importance for religion, although we cannot here
treat of it with the proper fulness and minuteness. 
It is the idea of \textit{a suffering Messiah}.
The thought of a Messiah had of course been
familiar for centuries, but nothing could be
more misleading than to suppose that Jesus
\marginpar{32}
Christ simply took over the prevailing view of
His day and country. He struck into a completely 
new line. Till then it had been believed
---and the belief is still a synonym for worldliness---
that the way to true sovereignty is brute
force. One has only to glance at Babylonian
sculpture to realize the brutal notion of lordship
or supremacy which prevailed in the ancient
world. Force, it appeared, was the secret
of majesty and power. Did Christ therefore
utterly reject the wish for power? Far from
it. Instead, as the author of \textit{Ecce Homo} has
expressed it, He laid claim ``persistently, with
the calmness of entire conviction, in opposition
to the whole religious world, in spite of the
offence which His own followers conceived, to
a dominion more transcendent, more universal,
more complete, than the most delirious votary
of glory ever aspired to in his dreams.'' He
claimed to be King, Master, and Judge of men.
He claimed this; but also He adopted the
unheard-of plan of maintaining, not in theory
only but in practice, that true power comes by
sacrifice and pain, and for His kingly portion He
\marginpar{33}
chose the Cross. This, as I have said, was a
gloriously new conception . No one had imagined
it before; but obviously when once it is under
stood, it puts us right up against an insistent, 
stupendous problem. Where does the
problem lie?

Some pages back we saw reason to believe
that the person of the Messiah was the central
fact in history, at once the pivot and the climax
of the Divine world-plan; and now we are
faced by the startling circumstance that---according 
to Jesus' self-description---this Messiah
is to perish in a death of shame. He is to die
thus mysteriously notwithstanding His incomparable 
greatness and innocence of life. How
can this be? It need scarcely be pointed out
that ordinary analogies between Jesus and
ourselves are here of no avail. No general
principles will suffice, and of this we are conscious
in our best moments. We men and women
have contributed to the world's sin; therefore,
as we shall all concede, justice prescribes for us a
personal share in the world's pain. Yet in the
case before us, so exceptional, so unique in
\marginpar{34}
moral majesty and self-abnegation, in a life
where sin has no part or lot, there is appointed
a death of unexampled contumely and suffering.
How can it be explained? The answer to this
tremendous problem surely lies in a direction to
which the Sufferer Himself has pointed. There
are two great passages in the Gospels in which
our Lord's teaching on His death is recorded
with entire clearness. The first is St. Mark x. 45:
''Even the Son of Man came not to be ministered
unto but to minister, and to give His life a
ransom for many.'' The other is St . Mark xiv. 24,
where, as He gave the cup in the Last Supper
to the disciples, He said: ``This is My blood of
the covenant, which is shed for many.'' In
both cases our Lord is referring to the forgiveness 
of sins, and what He declares plainly is
that this unspeakable blessing will be gained
for men at the cost of His life. If we are pardoned, 
we owe it to the death of Christ. His
death, in other words, had reference to sin. Just
because He was Messiah, the Deliverer sent of
God, He must take upon Him to deliver man from
the sorest of all troubles. He could not bear to
\marginpar{35}
pass by on the other side. How or when God
revealed to Him that this self-identification
with sinful men would lead Him to the cross,
is far from easy to determine, nor indeed is it
essential. But we know that His soul fed upon
the Old Testament; and this being so, it is
natural to think of the wonderful 53\textsuperscript{rd} chapter
of Isaiah as having been to Jesus the word of
God calling Him to His vicarious Passion.
However that may be, and whatever the avenue
by which He travelled, at all events He spoke
the words just cited in the full, clear certitude
that He must stoop to conquer; that only as
''lifted up'' by crucifixion could He draw all
men to Him. So that what Christ leaves on
our mind, as we ought to note emphatically,
what He leaves there as the central fact
of the world, is \textit{the Messiah dying for sin}. It
is a picture and a fact which every serious man
must gaze upon with all his soul and mind and
strength. It is the supreme reality of human
life. And it means at least this, that whereas
self-consciousness in us is one of the gravest
moral faults which separates us from others
\marginpar{36}
as by a yawning chasm, the great self-consciousness 
of Christ drew Him so close to us that
at the last His love bore our sins in death.

Of the two parts into which our present
inquiry is divided, Jesus' Messiahship and His
Divine Sonship, the first is now completed.
We have learnt the opinion held by Jesus regarding 
His own mission, and we have inquired as
to the bearing of His Messianic position on such
things as forgiveness and the working of miracles.
Let us now ask what light is cast by Jesus on
\textit{His personal relationship to God.}

As a preliminary we may remark that, if it
should appear that Christ claimed to stand in
a unique and incomparable relationship to God,
this, in view of our former results, will be felt
by reasonable men as affording relief from the
gravest moral and intellectual difficulties. No
doubt it seems at first only to create a fresh
difficulty, since everything unique has a strong
presumption against it and requires more than
usually convincing proof. From a still higher
point of view, however, it is obvious that Jesus'
special Sonship mitigates the difficulties we have
\marginpar{37}
already felt in other portions of His teaching.
If He stands on God's side, addressing us in
God's name, it is not wonderful that He should
speak in tones of moral authority, or exercise
the prerogative of pardon, or present Himself as
bearing the world's sin in death. We can see
a meaning stealing into these facts, which fall
into a transparent order and fitness if we view
them in the light of His higher consciousness
of an unshared connexion with the Father.

Now that is precisely the word we want, the
word ``presuppose''; it describes more accurately 
than any other the real attitude of our
Lord. He did not dwell upon His Sonship---
I mean generally, in the Synoptic narrative;
He did not make it the explicit subject of
debate or argument. He assumed it rather,
in word and look. But what people assume
is just what they are surest of. It leaves
the deepest mark on the mind of the
observer, for what is done with a quiet
deliberation and composure always is done
with emphasis. It is thus that Jesus acts.
Gradually the disciples became aware that He
\marginpar{38}
was taking a place beside God, a place in which
He could have neither substitute nor partner.
His attitude meant that He was the Person
on Whom everything in religion turned, completely 
covering and determining our relation
to God. He is \textit{the} Son distinctively; and to
men He offers power to become sons of God
through His mediation. In the Son the Father
is revealed; and as there is but one Father, and
cannot be more, so there cannot be more than
one Son, supreme and absolute. ``Sympathy,''
it has been said, ``is not more a characteristic 
of Jesus than aloofness or reserve. How
ever fraternal His relations with others, they
were penetrated with this quality of separateness
and authority.'' Both aspects of the total
fact must be recognized. Never was there
a more loving heart than Jesus, Who is the
Elder Brother of us all; yet nowhere, not
once in all His life, do we find Him stepping
down and standing simply at our side. He
speaks freely of ``your Father,'' ``the Father,''
''My Father,'' and in a memorable scene recorded 
by the fourth Evangelist, combining
\marginpar{39}
both modes of designation, He employs the
double phrase, ``My Father and your Father,''
in which the distinction is sustained firmly.
And yet---while it is from His blessed lips we
have learnt that Father is God's name---
He avoids the phrase ``\textit{our} Father'' with a
care and (as it seems) a solemnity of omission
which can scarcely have been accidental. Not
only so, but in one of the best accredited parts
of the tradition He is recorded to have said,
''No one knoweth the Son, save the Father;
neither doth any know the Father, save the Son,
and he to whomsoever the Son willeth to reveal
Him''
(St. Matt. xi. 27). I have never heard
these words read aloud in public assembly, but
they brought a hush upon the audience, so lofty
are they, so ultimate, so inimitable and august.
% Sic. A period is missing at the end of this last sentence.

This unique Sonship, it is clear from the
Synoptic Gospels, formed the basis and inspiration 
of Jesus' life-work. It was in the
strength of it, and as commissioned and
authorized by it, that He accomplished His
redeeming service for mankind. He came, as we
have already seen, to set up the Kingdom of
\marginpar{40}
God's almighty and righteous love. Its establishment 
was the appointed task of the Messiah.
But---and this a point of first-rate significance---Christ 
knew Himself to be Messiah because
deeper even than Messiah He was the Son of
God. In that unshared filial life He knows
God, and is known of Him, in a mode which
admits of no kind of comparison with other
men. They are the lost children, who need the
Kingdom; He is the Son Who brings it in.
In the secret place of the inmost self-consciousness, 
in the sanctuary of personal feeling, He
experiences His filial unity with the Father.
And therefore---to repeat it yet once more---
He is sure of His equipment for the great
mission. To know Himself as Son is also, and
simply by itself, to know Himself called to make
the Kingdom a reality within the world of men,
to lay its eternal foundations by bringing home
to men at once the Father's holy condemnation
of sin and His compassionate mercy for the
sinful. As the only begotten Son of God, He
and He alone was able to lead lost sons back
to the Father.

\marginpar{41}
That all these singular professions regarding
His own Person must have left a deep mark,
cannot reasonably be doubted. Especially is
it clear that they could not but affect, and
affect profoundly, the minds and thoughts of
His disciples. And when we open the New
Testament, we find that it was so. No one
can say ``He that findeth His life shall
lose it, but He that loseth his life for My sake
shall find it''---no one can say such words, I
repeat, without having to take the consequences;
and in the case of Jesus Christ this meant that
men began to trust Him with the trust they
gave to God. Jesus saw this; He wrought for
it; He expected it; and when at last it came,
He joyfully gave God thanks. The problem
now remaining on our minds, therefore, is
whether He was equal to the place which He
had thus taken by accepting the religious faith
and loyalty tendered by His followers. He
had presented Himself as able to save to the
uttermost. And now that men turned to Him
with an honest and pathetic readiness to be
saved, had He the power to fulfil His chosen
\marginpar{42}
task? He had spoken words of eternal life,
and had connected them vitally with His own
Person. Was He able to make these words
good in the experience of believing hearts?
Was He able to do for them exceeding abundantly 
above all that they could ask or think?
To this question we now turn.


\chapter{The Christ of Experience}
\markboth{The Person of Jesus Christ}{The Person of Jesus Christ}

\textsc{In} our study of the historic Jesus it has
been made clear that He entertained a certain
view of His own Person, and put Himself
forward in a quite definite religious character.
He put Himself forward in the specific character
of Messiah---as the Deliverer, that is, sent of
God to rescue man from all his sorest troubles
and to bring in the new order which should
fully express Almighty Love. Furthermore, He
claimed to be equal to this task because in
some lonely sense, and by the constitution
of His being, He was the Son of God. Or, to
put it otherwise, He came forth professing to
be a Saviour, on the largest scale. We have
now to inquire more closely whether, and how,
this tremendous claim has been vindicated in
human lives. Is it or is it not the fact that
\marginpar{44}
Jesus Christ has exerted what we must call a
redeeming influence on men like ourselves? If
He has, what light is thereby cast upon His
Person?

It would of course be no better than
affectation were we, even for the temporary
purposes of argument, to regard it as an
open question whether Christ does or does not
save men. That He has enabled sinners to live
in fellowship with God, assured them of Divine
pardon, and inspired them with triumphant
moral power, will probably not be denied except
by those who consider spiritual experience as a
whole to be illusory. For Christians, however,
who cannot take this line, the redeeming might
of Jesus Christ is an assured and fundamental
fact; the point, indeed, is one on which they
are not at liberty to pretend ignorance. They
would not venture to call themselves Christians
unless they felt free, or rather felt bound, to
utter before Him the great testimony of St.
Peter: ``Thou hast the words of eternal life.''
All this, I need hardly say, is compatible with
widely-ranging differences of opinion as to the
\marginpar{45}
mode in which the salvation due to Christ has
been effected. We must not confuse redemption
as an experience with theories of its possibility.
At the same time, this underlying conviction
that Christ does redeem men, if held now in the
foreground of our minds, will help to safeguard
us from treating the present subject---our
Lord's place in experience---as only an imaginary
hypothesis, a curious or piquant problem on
which to sharpen our wits. It is a subduing
thought that all round us, at this very hour,
men are being saved by Christ.

Let us realize, then, that redemption by
Jesus is a fact, which we assume but do not
prove. It is there, confronting us with
inexpugnable reality before ever we proceed
to analyze or explain it; our only task,
accordingly, is to ascertain precisely what it
consists in, and to what high issues it moves.
Its actuality in this experimental sense, however, 
proves at the very outset that Jesus's
witness to Himself, as already considered, was
neither rash nor baseless. His promise is seen
to be well-grounded in reality. The profession
\marginpar{46}
of Redeemership made by Him, and on the
other hand the human experience that He
redeems, appear like the curves of a noble
arch rising up in lofty sweep to meet and join.

Time would fail were we to expatiate at
length on what Jesus Christ is known to have
accomplished in those lives which have received
Him by obedient trust. That is an unending
story. We are all aware that if any truth resides
in the higher human testimony; unless people
have conspired strangely to talk cant, without
any concrete or intelligible motive but in many
instances with the sole result of incurring loss,
persecution, and even death itself, Christ has
transformed their lives. Men and women like
ourselves have been re-created by His influence,
changed in the depths and inmost secrets of
being. In every man that change takes a
different, because a personal, shape. His
redemption is as original and individual a fact
as the colour of his eyes. Each rising sun,
touching the wing of sleeping birds, wakes
over the woods a fresh burst of melody, as if
the sun had never risen before; and just so,
\marginpar{47}
wherever a man finds and grasps redemption,
faith in the heart is a new creation, as if he were
the first to discover Jesus. Nevertheless, since
human nature is after all a unity, through
all this wonderful and incalculable variety there
run certain well-marked lines of resemblance,
certain uniformities of response to Christ and of
benefit received from Him. Let us select one
or two of these for closer scrutiny. We shall not
exhaust the subject, but we may hope to see
how inexhaustible it is.

First, then, as a cardinal certainty, we take
\textit{the felt presence of Jesus Christ with men}.
Since our Lord lived in Palestine, there has
been an innumerable company of believers,
who are sensible that He is theirs with so
intimate a nearness that they can hold fellow
ship with Him, can really possess Him as
an indwelling and controlling life. The late
Dr. Jowett, of Balliol, who held no brief for
orthodoxy, speaks in language of haunting
beauty on this subject. He points to ``the
knowledge and love of Christ, by which men
pass out of themselves to make their will His
\marginpar{48}
and His theirs, the consciousness of Him in
their thoughts and actions, communion with
Him, and trust in Him. Of every act of kindness 
or good which they do to others His life
is the type; of every act of devotion or self-denial
His death is the type; of every act of
faith His resurrection is the type. \textit{And often
they walk with Him on earth, not in a figure only,
and find Him near them, not in a figure only,
in the valley of death.} They experience from
Him the same kind of support as from the
sympathy and communion of an earthly friend.
That friend is also a Divine power.''

It may be there is something in this language
which goes beyond the experience of many
Christians. Yet, on the other hand, when men
speak of fellowship with Christ, be it in living
or in dying, they are not using highly-coloured
metaphors; they are not indulging in the
impatient hyperbole we employ so often when
excited or hard pressed in argument; they
are merely and simply reporting one of the most
real elements of their personal existence. They
mean a Presence, unseen yet unknown, which
\marginpar{49}
impinges on their lives day after day and hour
by hour; a Presence which, if they wished to
be rid of it, they would have to exert force
to thrust away. To this there is no proper
analogy elsewhere. Doubtless it is often said
that the spirit of Lord Salisbury or Mr.
Gladstone still abides with the great political 
party of which each was acknowledged
leader; but whatever be the truth in this form
of expression, it is used, quite certainly, with
a clear consciousness of its fundamentally figurative 
character. Thus instinctively we speak of
these great statesmen as \textit{departed}; but Christians 
of the type I have referred to could not consent 
to speak of a departed Lord. Again, when we
remember dear familiar friends now with God,
we remember them as they were, in their form
and habit as they once lived; but the Christ
with Whom believing men hold communion now
is not merely the Jesus who walked in Palestine;
He is the exalted Lord, present with His people
in the sovereign power of His resurrection and
as inhabiting a higher order than that of time
and space. And once more, we do not feel
\marginpar{50}
that anything in the present influence of our
departed friends, and our response to it, is
determining our relation to God. Yet this
precisely is what we feel in regard to Christ.
Our attitude to Him, and His unimaginable
love to us, affects our relation to the Father
in a profound and decisive manner. He gives
to us the life of God; He constantly renews,
sustains, and augments it.

From the beginning until now, there have
been those who denied this unseen but real
presence of Christ. A man may quite sincerely 
say, ``I am at a loss to understand,
when you speak of Christ's continual nearness 
to us; for myself, I am unconscious of
anything of that sort.'' Plainly, however, this
objection may be taken from various points of
view. Thus the objector may not himself
claim to be a Christian. In that case, it will
probably be agreed his position need cause no
surprise, since only Christians can have the
authentic Christian experience. It would,
indeed, be surprising were it otherwise; the
really disconcerting and unintelligible thing
\marginpar{51}
would be to find that a man could actually
have the Christian experience without wishing
for it, or even knowing it---like measles.
Not only so; but though he may be a true
Christian, it does not follow that he will have
realized at the very outset the deepest and richest
elements in Christianity. Elsewhere in human
life, certainly, we allow for wide margins of
nobler attainment. When a boy wakens to
the beauty and the charm of Nature, do we
suppose that \textit{at once} her sublimer secrets will
unfold, that at once he will understand the
lines---
\begin{verse}
\small ``Two voices are there; one is of the sea. \\
One of the mountains; each a mighty voice''?
\end{verse}
Could one whose sense of poetic power had
been faintly stirred by Scott's \textit{Marmion} claim
to appreciate from the very outset all that
Shakespeare, Milton, Wordsworth, have done
for men? In such cases we refuse, and
rightly refuse, to pare down the significance of
the greatest things to some poor minimum or
insipid average; for we are conscious that
within the infinite experience of poetic feeling
\marginpar{52}
abundant room is given for growth, enrichment, 
expansion. There is a progress from
more to more, as men ``follow on to know.''
Similarly in the religious field there are degrees
in our appreciation of redemption; and every
one who clearly recalls his own past is aware
that, if he has honestly given the Gospel a chance,
there has been recognizable though intermittent
progress in his certainty as to the greatness
and the love of Jesus Christ. We must not
then too hastily conclude that a conception
like personal fellowship with Christ is an
imaginative but unreal addition to the original
simplicities of mere obedience to His commands. 
Not only does it remain a problem
whether we \textit{can} keep His commandments
save as united to Him spiritually; but it is
an obviously just principle that the question
how much the Gospel offers us is to be answered,
not by a scrutiny of the partial attainments or
discoveries we have so far made, but by consideration 
of the promises held forth by Christ, as well
as the believing experience of past ages. In any
case, let us decline to measure the potencies of
\marginpar{53}
the Christian life by the meagrest and least
daring standard. Let us hope for nothing
lower than the Best from the God and Father
of Jesus.

No one can read modern literature on the
origins of Christianity without recognizing that
in a certain type of book this thought of Christ's
unseen presence is wholly lacking. What we
find, rather, is an attempt to put Jesus back
firmly into the first century, hold Him a prisoner
there, and draw a line round Him (as it were)
beyond which His personal activities must not
be permitted to extend. Is He more than
a dead Jew, who perished about 30 \textsc{A.D.}?
Now, when we look away from books to actual
life, we discover that Christ remains past \textit{only
as long as He is not faced in the light
of conscience.} So long as we bring into play
our intellect merely, or the reconstructive fancy
of the historian, He is still far off; we
need not even hold Him at arm's length; He
is not close to us at all. The change comes
when we take up the moral issue. If we turn
to Him as men keen to gain the righteous,
\marginpar{54}
overcoming life, but conscious so far of failure,
instantly He steps forward out of the
page of history, a tremendous and exacting
reality. We cannot read His greatest words,
whether of command or promise, without
feeling, as it has been put, that ``He not only
said these things to men in Palestine, but is
saying them to ourselves now.'' He gets home
to our conscience in so direct a fashion---
even when we do not wish to have anything
to do with Him---that we feel and touch Him
as a present fact. Like any other fact, He
can of course be kept out of our mind by the
withdrawal of attention. But once He has
obtained entrance, and, having entered, has
shown us all things that ever we did, He moves
imperiously out of the distant years into
the commanding place in consciousness now
and here. We cross the watershed, in fact,
between a merely past and a present Christ,
when we have courage to ask, not only what
we think of Him, but what He thinks of us.
For that is to bring the question under the
light of conscience, with the result that His
\marginpar{55}
actual moral supremacy, His piercing judgment
of our lives, now becomes the one absorbing
fact. His eyes seem to follow us, like those of
a great portrait. When men accept or reject
Him, they do so to His face.

But more. Do we sufficiently realize the
master force which has sustained the saints of
God in their darkest hours? Take the Christian
workers in our slums, in the rookeries of our
large towns; take the missionaries in Uganda or
Manchuria or the far South Seas. What power
enables them to endure not with persistence
merely, but with cheerfulness? Can we doubt
the answer? Think of those Uganda boys,
told of in the Life of Hannington, who, when
burned in martyrdom, praised Jesus in the
fire, ``singing,'' as the biographer has said,
''till their shrivelled tongues refused to form
the sound''---
\begin{verse}
\small
``Daily, daily, sing to Jesus,\\
\hspace{1em}Sing, my soul, His praises due,\\
All He does deserves our praises,\\
\hspace{1em}And our deep devotion too.''
\end{verse}
\marginpar{56}
Or take an incident like the following:---
The University of Glasgow conferred upon
David Livingstone the degree of Doctor of
Laws on his return after being in Africa sixteen 
years. The students, bent on fun, were
in the gallery, armed with sticks, pea-shooters,
and other instruments for assisting their natural
powers of making themselves disagreeable.
Livingstone appeared gaunt and wrinkled after
twenty-seven fevers, darkened by the sun, and
with an arm hanging useless, from a lion's
bite. The pea-shooters ceased firing, and all
felt instinctively that fun should not be poked
at such a man. Livingstone was allowed to
speak without interruption. He said that he
would go back to Africa to open fresh fields for
British commerce, to suppress the slave trade,
and to propagate the Gospel of Christ. He
referred proudly to the honourable careers of
many who had been with him in college, and
with sadness to the fate of some who had
gone wrong. ``Shall I tell you,'' he asked,
``what sustained me amidst the toil, and hard
ship, and loneliness of my exiled life? It was
\marginpar{57}
the promise, `Lo, I am with you always, even
unto the end.' The effect which the words
had, coming unexpectedly from one who was
both the witness and example of the promise,
could not have been surpassed since they were
first uttered in Galilee.''\footnote{Hardy, \textit{Doubt and Faith}, pp. 174--5.} 
% sic. Period missing after Galilee
The incident is
suggestive on another ground. It indicates
that the certainty of Christ's sustaining power
rests not on individual conviction merely, but
on the Lord's own promise. He \textit{undertook} to
give those who trusted Him this enduring
spiritual presence and power. As it has been
put in a well-known and exceptionally clear
sighted book: ``Jesus exerted a marvellous
spiritual influence by His personality during
His life, but, as that earthly life was drawing
to its close, we do not find Him contemplating
the withdrawal or diminution of that influence.
The very contrary. He promised its persistence 
and even its augmentation. That very
spirit with which He had baptized men, and
which it only too inevitably seemed must pass
with His earthly presence, is the very thing
\marginpar{58}
which most impressively, He declared would be
given more than ever. By this spirit, He
clearly meant certainly nothing less than all
that His present personality had been; and
indeed, His meaning He often simply expressed
by saying that He---all that the personal contact 
with Himself had meant---would not pass.
It is this note which is the most remarkable
characteristic of the latter phases of the utterances 
of Jesus. There is nothing like it in
the later teaching of any other man.''\footnote{Simpson, \textit{The Fact of Christ}, pp. 73--4.}
% sic. This footnote lacks a marking in the text, presumably it belongs to this quote.
The
words of Christ, then, echoed by our experience
of their fulfilment, prove that belief in His
constant presence is no fiction.

It is of course an easy thing to ask difficult
philosophic questions regarding the abiding
nearness of the Lord. A child of three will
often ask questions about God and man, heaven
and earth, which no living man can solve;
when the mind is dealing with an infinite
object, it will always be so. But these subtler
problems lie beyond our present aim. We are
merely registering the experience of Christian
\marginpar{59}
men; and it is a mere fact that at this hour
there are thousands to whom the felt presence
of Christ is as real as the consciousness of
right and wrong. Surely we cannot refrain
from seeking an explanation of this extra
ordinary Person who is still close to His disciples. 
Who is He? Whence has He come?
How is it that death did not silence and remove
Him as it has silenced all the rest? We have
no choice but to try and clear up our minds.
When the Church did that, she made the
Creeds; and we well know where she set Him
in the great confessions of her faith. If any
one objects to Creeds, there is no reason at all
why they should not be put aside for the time
being, provided we replace them by two books
which make a tolerably good substitute---the
New Testament and the hymn-book. The
simple fact that New Testament believers
prayed to Christ sufficiently demonstrates what
\textit{they} held true regarding His personal spiritual
presence. And when we scan a great Christian
hymn like ``Jesu, Lover of my soul,'' the witness
borne by what has been called ``the layman's
\marginpar{60}
manual of theology''---the hymn-book---stands
out with the same decisive evidence.

Let us now turn to \textit{the conquest of sin attained
through Christ}. As an element in experience
this is as indubitable as the other, and it lights
up His Person no less strikingly.

Sin is conquered in two ways. It is conquered 
first and foremost when God destroys
its power to exclude us from His fellowship;
or, in plain English, its back is broken when
we know ourselves forgiven and thus gain a
great initial assurance of the Divine love which
enables us to make a start in the Christian life
and to do something like justice to the Gospel.
Once we know that God is ours, and that He
pledges Himself to keep us His, we can put
up a good fight; for now sin is under our feet,
and what remains is only that by God's help
we should steadily crush its life out. This is
essential to all true and triumphant conflict.
Where do we receive this impression of forgiveness, 
on which everything depends? Men do not
gather it out of the air. It is as far as possible
from being a commonplace. On the contrary,
\marginpar{61}
it has come to all who now possess it in a quite
specific way; it has come to them in the presence
of Jesus Christ and very specially in the presence
of His Cross. For there we confront the full
expression of God's mind both to sinners and
to sin.

Consider the man who is standing before the
Cross, with soul laid open in humility to its impression. 
What does he feel? Two things
certainly. First, he feels that sin is condemned
there---condemned absolutely, for good and all.
Place yourself before the dying Christ, and at
once you become aware that through Jesus'
eyes, as we behold His death, there looks out
upon us, with humbling and convicting power,
the very holiness of God with which evil
cannot dwell. Never was sin so exposed, and,
by exposure, so doomed, reprobated, sentenced,
as by His treatment of it from the beginning to
the end. When Christ had done with sin, it stood
there a beaten, powerless thing; paralyzed, vanquished, 
dethroned, stripped of every covering,
every mask, flung out in utter degradation. Now
as we feel His look upon us, under the shadow of
\marginpar{62}
the Cross, He is doing this still, doing it to \textit{us}.
The voice of His Passion condemns our evil;
but in its unheard tones there is audible the
voice of God. In virtue of His oneness with
the Father, Christ declares and brings home to
conscience the final truth regarding the sinfulness 
of sin. He forgives our trespass only
because in God's name and with God's authority
He has first passed judgment on it from which
there is no appeal.

This is the first strain we may distinguish in
feeling, but it is not alone. Beside it, or
rather interwoven and suffused with it, is the
feeling also that there is love in the Cross;
love beyond all we could ask or think. It was
for love that Jesus died. And let us not miss
the wonder of the circumstance that this dying
love is felt as \textit{the love of God Himself}. Actually,
literally, and just as we experience it, it is the love
of the Eternal. It is God Who in Jesus meets us,
evoking faith, calming fear, cleansing conscience;
giving us, as Bunyan puts it, ``rest by His
sorrow and life by His death.'' If the Cross means
redemption, then it is by God Himself and none
\marginpar{63}
other that the price of redemption has been paid.
In what He undergoes on Calvary Jesus is not
merely pointing upward to a Divine love
beyond and above His own person, a love
which He does no more than announce; He is
bringing it in upon our soul. He puts it in
our hand as we survey the despised shame, and
as we gaze on Him there comes home to us the
inexpressible pathos and sacrifice of the words:
''He that spared not His own Son, but delivered
Him up for us all.'' The passion of God is there.
When we drop the sounding line in that sea, we
hear the lead plunge down into unfathomable
waters.

Is it too much to say that Jesus' dying love
is itself the love of God? Surely not. Let us
ask why the Cross of Christ does not revolt us.
Why, in view of that ineffable Passion, do we
not cry shame against the government of the
world? Why is it not felt as the most insuperable 
of difficulties by all who attempt to justify
the ways of God to men? For here is the
best and holiest Soul of history, whom we dare
not praise because He is above all praise;
\marginpar{64}
yet His career and His end are such that we
still name Him ``the Man of Sorrows.'' Why
do our hearts not flame with indignation against
God Himself that this should have been
Christ's appointed lot? Because we feel, even
if it be dimly, that the love which meets us
there and endures all for our sake is veritably
the personal love of God. Christ is not a good
man merely, Whom God seized and made
an example of for all time; in His life, rather,
the Love that is supreme has stooped down to
suffer in behalf of men. This and nothing else
has broken the world's hard heart. What
might and must have been the worst of perplexities 
is all transcended, if we but catch
the pure shining in it of the Divine mercy in
such an intensity of revelation as solves all
difficulties and calms all fears. Now, as Luther
said, ``we have a gracious God.'' The Cross
is a casement opening on a new world.

\begin{verse}
\small 
''The very God! think, Abib; dost thou think? \\
So, the All-Great, were the All-Loving too---\\
So, through the thunder comes a human voice\\
Saying, `O heart I made, a heart beats here I\\
\marginpar{65}
Face, my hands fashioned, see it in myself!\\
Thou hast no power nor mayst conceive of mine,\\
But love I gave thee, with myself to love,\\
And thou must love me who have died for thee.'''
\end{verse}

So much, then, for forgiveness. The second
mode in which sin is overcome is by the breaking 
of its tyranny in character. Not only is
pardon mediated to us by a Christ Who loves
while He condemns---expiating sin that He may
be able to forgive it---but also it is Jesus Christ
Who gives power over evil habit. In the field
of religious experience there is no point
as to which so wide and joyous unanimity
prevails as in regard to the moral inspiration 
and triumphant energy that flow from
Christ to tempted men. People like us have
been saved by Him; saved not in a vague
or unverifiable sense, but saved from contempt, 
saved from despair, saved into freedom
to stop sinning, saved into the successful pursuit 
of goodness and likeness to the Father.
The glorious fact, thanks be to God, is being
repeated every day. Men who have lost faith in
aspiration, whose friends have given them up
\marginpar{66}
in sheer disgust or in sad weariness, encounter
something or some one that persuades them to
commit their lives to Jesus Christ; with what
effect? With this effect, that instantly or by
degrees new life is imparted to them, new
tastes, hopes, preferences, inclinations, motives,
delights; until not in boasting but for sheer
thankfulness they dare to say: ``I can do all
things through Christ that strengtheneth me.''
Christ keeps what we entrust to Him. I can still
hear the tones of Professor Henry Drummond's
voice, twenty years ago, in those wonderful
Edinburgh University meetings, as he explained 
what Christ would do for us. ``I
cannot guarantee,'' he would say, ``that the
stars will shine brighter when you leave this
hall to-night, or that when you wake to-morrow
a new world will open before you. But I do
guarantee that Christ will keep that which
you have committed to Him. He will keep
His promise, and you will find something real
and dependable to rely on and to lead you
away from documental evidence to Him Who
speaks to your heart at this moment.'' And
\marginpar{67}
it has come true, every word. In our time
it has come true as in the times before us,
exactly as Christ said. ``He that followeth
Me shall not walk in darkness, but shall have the
light of life.'' Always, everywhere, it is found
that those who answer His demand receive
His promise. The power of Jesus Christ to
produce and sustain character, then, is an experimental 
fact as well-grounded as the law of
gravitation. For those who cast themselves
on Him, in faith's great venture, accepting
honourably the conditions under which alone
spiritual truth can be verified, the truth becomes
luminous and certain. They discover that to
be Christians is not to repeat a creed, or to
narrow life into a groove; but to have a strong,
patient, divine Leader, whom they can trust
perfectly and love supremely, Who is always
drawing out in them their true nature and
making them resolve to be true to it through
the future; Who looks into their eyes when
they betray Him, making them ashamed;
Who imparts the forgiveness of sins and gives
power to live in fellowship with God. Apart
\marginpar{68}
from this, His call would only mean a new
despair. But His strength is made perfect
in weakness.

Is not all this the token of something unique
and superhuman in His Person? Could a man
who had perished in the first century inspire
us thus? Could he support us in the conflict
with self and evil by his sympathy and communion? 
Could he dwell within us, possessing
will and heart? Surely to think so is to play
with words. For we know what man can do,
and also what he cannot. If we search for
words to express the absolute unity of God
with man, we light inevitably on some
such words as we have employed regarding
Jesus Christ. If God were to come in person
by incarnation, by personal presence, are not
these the very signs, the authentic powers,
by which His glorious advent might be
known?

Finally, in Christ we have \textit{a perfect revelation
of God the Father}. This comes appropriately
at the point we have now reached, since it is
\marginpar{69}
always through redemption as an experience
that revelation is vouchsafed. Through Christ
the Saviour we see back into the Father's
heart from which He came.

Now we may lay hold of this principle by
either of two handles; it scarcely matters
which. Of course it is unquestionable that
men may find \textit{some} thought of God elsewhere
than in Christ. To deny that would be monstrous. 
For example, it was not reserved for
Jesus to reveal God as Creator or Sustainer
of the world; ages ere He came, these stupendous 
truths had become the possession of many
a heart, bringing light and calm. It was not
reserved for Jesus to make God known as
supreme Moral Authority; this conviction also
had been attained. It \textit{was} reserved for Him
to manifest God in the character of loving and
holy Fatherhood, \textit{a Fatherhood which embraces
all the world}. History teaches that men cannot
come, and never have come, to a distinct
impression of the Fatherhood of God, in the
loftiest and most subduing sense, save through
Jesus Christ. ``Neither doth any know \textit{the}
\marginpar{70}
\textit{Father}, save the Son''; ``No man cometh
unto \textit{the Father} but by Me''---these august
words are confirmed by the facts of life. But
not in words merely was the great revelation
given. The life of Jesus, as it moved onward,
was a ceaseless proclamation of the novel
thought of God. For the first time it was
shown how God loves each individual life,
seeking the lost untiringly and counting no
price too great for their recovery. Nowhere
does Jesus take this message into the atmosphere 
of theory; He is content that it should
rest in its own unity, as if any analysis must
disturb its beauty and its power. But He
wrote the fact in actions which could never
be forgotten. When His life was over, there
were men and women who knew that God
was just like Jesus---as loving, as holy, as full
of saving and transforming power; knew, too,
that at once, and before we become any
better, He is willing to be our Father. In
the exquisite words of R. W. Barbour, ``At
every step of Christ's life He let loose another
secret of God's love. All God's love is in
\marginpar{71}
Christ. Think of every act, every event, every
incident from His cradle to His grave, and
you will find the Father's love stealing out
somewhere.'' But, he goes on, ``God's love
must be measured by the \textit{whole} work of Christ.''
In other words---remove the Cross, and at once
the revelation is lowered and impaired. For
the last and highest truth about God was
uttered silently at Calvary. That was the
final chapter in the Son's exposition of the
Father. By this death for sinners Jesus unveiled 
a Fatherhood of such dimensions---such
breadth and length and depth and 
height---\textit{that sacrifice comes into it at last}. The past
of Old Testament religion was indeed rich in
pictures and promises; yet never had grace
been dreamt of so infinite as to die for man.
Prophets had spoken of redemption, but the
cost of redemption to God lay hid until He
came upon whom it was to fall. Till Christ
had been here, had come and lived and
suffered, it was not known to any single
soul that God loves all men---whoever
they are and whatever they have done---nay 
\marginpar{72}
more, that God has vindicated the reality
and passion of His love by the endurance of
vicarious pain. Nor is it until Jesus has
entered our experience, revealing the blessed
love of God and His communion with the
sinful, that you and I can have hold of the
Divine Fatherhood in a way that makes it
real, near, and sure to our minds.

Some years ago the question was asked me
by an earnest and able man: ``What need have
we of Christ? The religion of the Psalms,''
he added, ``is enough for me.'' It was curiously
difficult to answer him. If a man honestly
feels that the faith expressed in the
twenty-third Psalm, or the fifty-first, or the
hundred-and-third, quite satisfies his conscience 
and mind and heart---how shall we
then present Jesus Christ as the medium of
\textit{a new and essential gift}? Part of the answer,
doubtless, is that, although we may be
unable to anticipate Christ as sheerly
indispensable, yet we are so made that
at once, when we have beheld Him, we
know without reasoning that He is necessary.
\marginpar{73}
Besides that, it may be pointed out that
the Psalmists themselves are aware of deep
longings and desires which their own
religion could not wholly quench. With all
their priceless treasures, they yet lack something. 
And we may go further. More and more
men are conscious that we need a great fact,
a reality external to self and unchanging with
the centuries, on which we may ground our
confidence and trust in approaching the Holy
One with Whom we have to do. We need,
absolutely and always, a fact to which we
may simply respond, which is neither the hypothesis 
of our reasoning nor the creation of our
wishes nor the postulate of our reverent
hope, but, on the contrary, a substantial and
significant existence which confronts us as an
irrefragable element in history, and to which
our noblest aspirations and hopes can be
fastened. This reality, it is plain, must needs
be a Person; for only a Person can show
us the personal God. We turn then to
history, and there we encounter Jesus of
Nazareth as the Divine answer to our
\marginpar{74}
human longing. Thus our question is fairly
met. Do men need Christ? How is the
Father known? He is known only in the Son.
Only so are we \textit{sure} of a love that saves to the
uttermost, a God Who is faithfully and unchangeably 
Redeemer.

But we may view the revelation of the
Father otherwise. Try to think out carefully
what you mean by God. Your mind turns
immediately (does it not?) to the Divine
character---the holiness, the love, the power,
the eternal grace to sinners. Now when you
put this down---combining it in a spiritual
unity---and turn next to the picture of
Christ, you discover that almost without your
being aware a strange thing has occurred.
Instinctively you have transferred to God
those personal features, qualities, and characteristics 
which appear in Christ. The
attributes of the Christian God, in short, are
but the traits of Jesus' character exalted to
infinity. Without knowing it, certainly with
out intending it, you have verified the Lord's
\marginpar{75}
own saying: ``He that hath seen Me hath
seen the Father.''

But obviously this at once creates a vast
problem for the mind. Can one reveal God perfectly 
save He who \textit{is} what He reveals? Apart
from such real identity or unity between Revealer
and Revealed, must there not be discrepancy
somewhere in the revelation---an aberration or
refraction of truth for which in the end we
must still sadly make allowance? Prophets
speak but a fragment of God's mind, for they
are messengers only. But He who perfectly
declares \textit{the Father}---what is \textit{His} place and
position? No better name surely is imaginable
than that which He bears throughout the
writings of the New Testament---the Son of
God. In will, in character, in redeeming power,
He is one in person with the Eternal whose
being He unfolded to the world.

Christ saves, yet only God can save. There,
in a simple and elementary reflection, lies the
original but also the permanent foundation of
a great thought which men naturally have felt
so hard---the Divinity of Christ. It looks
\marginpar{76}
the merest mythology; in fact, it is but a
transcript of experience. A man tries the great
venture of faith. He has studied Christ in
all the books. He has sat still and thought
and tried to see through his thought the very
face of Christ whom he longs to understand;
and he has not succeeded. Then he rises up
for action and resolves to seek the illumination
of obedience. And more and more as he
goes his way, doing the duty, bearing the
burden, always with his eye upon the Leader,
it dawns on him that in the new life God
and Christ are morally indistinguishable. To
believe in Christ, always, is to believe in
God. To do Christ's will is to do God's will.
And secretly, in the hour of meditation, when
we try to look into God's face, still it is the face
of Christ that comes up before us. Now the
man to whom this happens, if he puts intelligence 
into his faith, must needs raise the last
and final question---Who is this marvellous
Personality? Is He but an incident of history,
a wave rising on the sea of human life, as
the billows innumerable surge and melt in
\marginpar{77}
mid-ocean? Is He the passing creature of time,
or has He not rather come forth out of the
uncreated life of the Eternal? Eternity or
time---do we have to choose between them?
What if Christ belongs to both at once! What
if He is as old as the saving love of God, yet
emerging into history at a definite spot in the
long past! It is a paradox, of course; yet
truth, we all know, may be stranger far than
fiction.

\chapter{Jesus Christ and God}
\markboth{The Person of Jesus Christ}{The Person of Jesus Christ}

\textsc{We} have endeavoured in the preceding
chapters to determine the position occupied
by Jesus alike in His own message and in the
experience of His Church. Our conclusions, so
far as they went, were clear enough. His self-consciousness
was unique, and He assumed a
place in the relationship of God to man which
no other can ever fill. He presented Himself
explicitly as the Way to the Father. As the one
essential Way, to miss which is to miss every
thing worth calling life, He confronted those who
heard Him with the supreme moral problem
and exerted the supreme moral authority.
Advancing next to His reality in experience,
we sought to analyze and (as it were) tabulate
\marginpar{79}
what He has been to men and women who
by universal consent have made Christian
history. It is an infinite theme; but we
selected as fairly typical and characteristic
these three points---first, that Christ is still
present in full personal influence, so that
those who accept or reject Him do so to His
face; second, that in Him we attain the conquest 
of sin; and finally, that He imparts to
us a satisfying sense of God. In Him we see
the last reality of the universe as Holy and
Almighty Love.

It remains now that so far as may be we should
gather up our results and give them what
may be called their final meaning and direction.
We proceed to ask who this extraordinary
Person is, whence He came, and where in the
hierarchy of spiritual being He must be placed.
It is of course a long-debated point of theory;
but to me at least it is questionable whether
thoughtful disciples will ever consent to stand
before Jesus Christ in complete silence, vetoing
their own eager thought, asking no questions,
working out no answers, even while all the
\marginpar{80}
time they are sensible of His stupendous and
incomparable significance for religion. We have
really no option but to think about Him with
all our might and with the best intellectual
instruments at our command. Reason---which
is more than logic---insists on coming into our
faith. Nothing is easier, nothing is cheaper,
and, I believe, nothing in the long result is
more fatal, than to give men the impression
that our religion will not bear being thought
out to the end, that it dies if we bring it into
the sun. The absolute and final issues created
by Jesus must be faced. Now, if we regard Him
as Saviour, we must see Him at the centre of
all things. We must behold Him as the pivotal
and cardinal reality, round which all life and
history have moved. That is a place out of
which His Person simply cannot be kept. We
dare not permanently live in two mental worlds,
dividing the mind hopelessly against itself.
We cannot indulge one day the believing
view of things, for which Christ is all and
in all, and the next a view of philosophy or
science for which He is little or nothing or
\marginpar{81}
in any case ranks as quite subordinate and
negligible. After all we have but one mind,
which is at work both in our religion and
our science; and if Christ is veritably supreme
\textit{for faith}, He is of necessity supreme altogether 
and everywhere. Growingly it becomes
impossible to revert to a scientific or philosophic 
attitude in which the insight into His
central greatness which we attain in moments
of religious vision is resolutely and relentlessly
suppressed. At every point we must be true
to experience, and the deepest experience we
have is our experience as believing men. Hence,
if the thought of Christ we have reached is
valid, it must be carried consistently up to the
top and summit of being, as something which
is true with a truth that will Stand the closest
scrutiny and verification of sympathetic minds.
In this spirit let us inquire into the relation
of Christ to God. If His self-consciousness is
thus absolute, if He exerts this regenerating
power in experience, how shall we name Him
best? Who must He be in His proper self?

%CONTINUE HERE

Now it is worth while to accentuate the fact
\marginpar{82}
distinctly that at this point we are faced by
a very real alternative. For we Christians are
bound to place Christ either within the sphere
of the Divine or without. Either He is one
with the Father, or He somehow is different
and unlike. Take a concrete instance, probably 
not uncommon, among the noblest and
most magnanimous minds of our day. Here
is one whose experience of Christ may be described 
by saying that he has turned to Him for
strength to do the will of God, and the required
strength has been given. He has sought in
Him the forgiveness of sins, and in consequence
the oppressive load of guilt has been lifted off,
and he has obtained peace with God. Not
only so; but day after day he finds that the
intenser his affection for Christ and the more
whole-hearted his devotion, the more swift
his progress in the pursuit of righteousness.
Gradually he becomes assured that Christ is
superior even to conscience. While conscience
may err, and has often done so, he has now
learned that the will of Christ, when we are
certain that we know it, is entitled to
\marginpar{83}
control life from end to end. If then a man
has undergone these experiences, so thrilling
and so revolutionary, if he recognizes them
intuitively as forming part of life in its most
sublime and commanding aspect, what must be
his measured conviction regarding the Person
through whom they have been mediated? He
may say that he is prohibited by intellectual 
reasons from accepting the Divinity
of Christ. He finds the doctrine logically or
speculatively incredible. May we not suggest
to him, however, that his experience being what
it is such rejection can only mean that he is
illegitimately isolating his intellect from the rest
of life? He is declining to let his ultimate
decision be controlled by the best of his knowledge 
and his feeling. In moments of religious
vision we see deepest into the life of things
and grasp most firmly the solid pillars and bases
of reality; what in such high hours we know
to be true regarding Jesus Christ ought therefore 
to determine our permanent judgment on
His Person. And may we not further urge
on him that already---if our description of
\marginpar{84}
his experience be correct---he has accepted
the divinity of our Lord \textit{morally}, inasmuch as
he has accepted it by conscience, by personal
loyalty, by spiritual trust and self-committal.
In words of the late Dr. Dale which once read
can never be forgotten: ``When the reality
and greatness of His redemptive powers are
known by experience, a man will have no great
difficulty in believing, on the authority of the
words of our Lord in the Four Gospels, that
He will raise the dead and judge the world.
These spiritual relations to Christ receive their
intellectual interpretations in the doctrine of
His divinity. The doctrine is an empty form
where they are not present; and where they
are present the substance of the doctrine is
believed, though every theological statement
of it appears to be surrounded by difficulties
which make it incredible. It is an immense
gain for the intellect to receive and grasp
the doctrine; but the supreme thing is for
Christ to be really God to the affections, the
conscience, and the will. He whom I obey
as the supreme authority over my life, He whom
\marginpar{85}
I trust for the pardon of my sins, He to whom
I look for the power to live righteously, He to
whose final judgment I am looking for eternal
blessedness or eternal destruction, He, by what
ever name I may call Him, is my God. If
I attribute the name to another, I attribute to
Christ the reality for which the name stands:
and unless, for me, Christ is one with the
eternal, He is really above the eternal---has
diviner prerogatives and achieves diviner works.''

The last sentences of this extract remind us
that it is possible for a man to refuse to Christ
the supreme predicate of divinity, because he is
unconsciously operating with a one-sided or imperfectly 
ethical conception of the Divine. God---what 
meaning, after all, belongs to that great
word? What must its import be for \textit{Christians}?
It means, I think, Love, Holiness, and Power
in living combination and exalted to infinity.
But is it not just this unity of qualities which
we behold in Christ? Are not these precisely
the attributes in virtue of which He subdues
us to Himself, forgiving all our iniquities,
evoking our obedience, and elevating the soul
\marginpar{86}
triumphantly above the coward fear of things?
If then we define the term ``God'' in such a
way as to exclude Christ, this means---it can
only mean---that we make Him \textit{superior} to our
usual thoughts of Godhead. But if Jesus is
highest in the highest realm of which we have
any knowledge, then to speak of His Divinity
is not merely natural; it is forced upon us
if we wish to express our indebtedness to Him
for everything which can be called salvation.
So that to call Jesus God is, in Herrmann's
words, only to give Him His right name.

Let us try now to contemplate the matter
from a different point. Take the wonderful
conception---be it true or false, it at least is
wonderful---that in the Person of Christ the
Almighty God has Himself come amongst us,
has appeared in history ``for us men and for
our salvation.'' And regarding this conception,
let us ask the extremely practical and incisive
question: Do the people who have to live in
a world like this \textit{require} such a faith? Is it
a faith round which they can build up a
joyous and triumphant religious life? Do they
\marginpar{87}
need it to solace grief, to repel temptation, to
sustain endurance, to banish fear? In our
time nothing nobler has emerged, nothing more
Christlike and fraternal, than the steady fight
against useless pain. The sense of pity is
diffused widely, so that multitudes of people
steadily make a conscience of the curable
suffering of the world. They are mostly
agreed that the direst sorrows to be met with
are of a mental character. They concern the
spirit, not the body. When we examine our
selves, therefore, regarding our real ability to
offer deliverance from the worst grief and pain,
we find that the most poignant and paralyzing
dread of which our minds are capable is
uncertainty or darkness as to the love of God.
A generation since one of the most illustrious
English men of science wrote as follows regarding 
the all but unbearable suffering involved for
him in the surrender of faith in God: ``I am
not ashamed to confess that with this virtual
negation of God the universe to me has lost its
soul of loveliness; and although from hence
forth the precept to `work while it is day'
\marginpar{88}
will doubtless but gain an intensified force from
the terribly intensified meaning of the words
that `the night cometh when no man can
work,' yet when at times I think, as think at
times I must, of the appalling contrast between
the hallowed glory of that creed which once
was mine, and the lonely mystery of existence
as now I find it,---at such times I shall ever
feel it impossible to avoid the sharpest pang
of which my nature is susceptible.'' Total
eclipse of faith in God the Father---there is
nothing which cuts so deep as that, nothing
which so whelms the soul in impenetrable
gloom, nothing which can be compared, for
hopelessness, for weakness, for power secretly
to instil the dire conviction that all is vanity.
Many symptoms indicate that it is very wide
spread at the present hour. Certain of the
most sombre and powerful novels which have
been written and circulated by tens of thousands
in the last few years have precisely this for
motive---that the universe is a death-trap,
and that we men and women have been caught
helplessly in the trap by a Power too great
\marginpar{89}
for us to control, too callous for us to soften, too
far for us to reach, deaf to supplication, blind
to pain. Not long since I read the following
sentence in a novel of this sort at the close of a
protracted scene of fatalistic tragedy. ``In every
hour of every day and every night,'' said the
writer, ``uncounted human creatures writhe
like severed worms under the spade of chance.''
The author believed that. His book was
written to set it forth. What shall we say
to men who are in the grip of a pessimism so
dark, so unrelieved? When we speak of the
love of God, do they not answer, ``He loves
us, does He? What then has He done to prove
it? It were possible to believe in a God who
did something, but He does nothing. The
ages pass and He gives no sign.'' What have
we Christians to reply? Unless we \textit{can} reply,
be it remembered, we have really no Gospel.
The line we employ is too short for human need.

Nor need we imagine that the sharp edge
of the problem touches other people merely.
Very piercingly it comes home also to
ourselves. Can we forget those evil ways of
\marginpar{90}
which so often we are weary, while yet we have
no power to forsake them? Can we forget
those persistently and horribly cruel allurements 
which so often return upon us, torturing
and confounding our best desires, depriving us
of victories we had hoped to gain for good
and all, till in our frailty and anguish we
are fain to cry aloud that God cares nothing
for our rise or fall? Surely in our own lives,
though we may have escaped the fiercer outward 
suffering, each of us who knows the
conflict of temptation is aware that we too require
the assurance that the Power that made us
and placed us here indeed cares for us and is
acting in our checkered lot. How close in such
dark hours comes the fear lest God is far
away, too distant for succour, too great to
observe our bitter need. ``Many there be that
say of my soul, There is no help for him
in God.''

Let us take this vast and complex fact---the
fact of suffering, whether in body or in soul---
and let us insist on knowing how it may be
adequately met. How shall we assure men in
\marginpar{91}
their agony that God veritably is love? For
my part, I find the one completely satisfying
solution in the certainty that Christ, the Son
of God, has indeed suffered in our behalf.
As it has been put, ``The greatest fact in
the history of our world is that the Son of
God became one of ourselves, and lived and
died as God manifest in the flesh. Thus He
translated into our human speech the language
of the Eternal. He revealed in our human
conditions the inmost character of God. And
He did more than this, for He assured us, by
the surrender of Himself to humiliation and
death, that God did not regard His world with
callous indifference, but with deep compassion
and love. The message of the Incarnation
is that God loves us better than He loves
Himself.''

Here, then, is the watchword for a conquering 
faith---\textit{God was in Christ}. If---I confess
the ``if'' is a tremendous one---if a Divine
Person has been with us, living a human life,
working with human hands, weeping human
tears, bearing our load and carrying all our
\marginpar{92}
sorrows---then at once all things are changed.
For then we face life, whatever it may
bring of light or shadow, with hearts at rest.
How St. Paul leans on this truth in the closing
words of Romans viii.! All the dark things have
been coming back on him---tribulation, famine,
nakedness, peril, sword---the whole squadron of
evils striving (as it were) in unison to break
down his central confidence and shroud his
soul in darkness. How does he meet them?
By casting himself down into the depths of
the self-abnegating Divine love and staying his
heart on a revealed fact: ``He that spared not
His own Son but delivered Him up for us all.''
Here is what God has done. He \textit{did}, for love's
sake, all that is represented by a career in our
world ending upon a cross. In this is seen
the measure of a Father's love for His blinded
and dying children.

It is impossible to over-estimate the practical
significance and appeal of this great faith.
We cannot ever exhaust the power of an eternal
Love pledged to us in Christ's self-sacrifice.
With this certainty in our heart, we can enter
\marginpar{93}
the room of the tortured invalid, or the mother
mourning her dead child; we can sit by their
side and say: There is love for all, for you, in
God above, and what proves it is Jesus' life and
Jesus' death. Do not cease to grieve; in grief
there is no sin; but also do not believe that
even grief is unknown to God your Father.
This cup of pain you are drinking now, He also
drank; in all our afflictions He was afflicted.
Unless to those whom pain is breaking we can
offer this Gospel, this proclamation of the love
of Him who came in person and shared our low
condition, then we may well believe that in the
last resort the problem of humanity is too much
for our resources. There are fears we cannot
assuage, there are griefs we cannot solace, there is
darkness we cannot lighten, save by telling men in
the warm accents of personal sympathy of that
Divine mercy which did not refuse love's last
office, but stooped to suffer for the needy.
Words to express this triumphant creed you
may take from the New Testament or from
Browning. With St. John you may say, ``Herein
is love, not that we loved God, but that He
\marginpar{94}
loved us and sent His Son.'' Or you may choose
the modern poet's lines:
\begin{verse}
\small
``What lacks, then, of perfection fit for God\\
But just the instance which this tale supplies\\ 
Of love without a limit? So is strength,\\
So is intelligence; let love be so,\\
Unlimited in its self-sacrifice;\\
Then is the tale true and God shows complete.''
\end{verse}
God is no remote Deity, watching from afar a
stricken world; He is a Presence and a Redeemer
in our midst.

Such a train of reflection obviously dissipates 
one charge which has not infrequently been made
against the doctrine of Christ's divinity. It is
gravely reproached with being a scholastic notion,
interesting to metaphysically minded persons,
but in no special connexion with practical
and effective life. In all seriousness, however,
can there be anything more important for
life and practice than to have borne in
upon us an overwhelming and sublime impression 
of God's love? I fancy that as we grow
older, as we think longer and work harder
and learn to sympathize more intelligently,
\marginpar{95}
the one thing we long to be able to pass on to
men is a vast commanding sense of the grace
of the Eternal. Compared with that, all else
is but the small dust of the balance. Look
at the noblest workers, in the home Church,
look at our missionaries over all the world;
what is the inward conviction which enables
them thus to
\begin{verse}
\small
``Set up a mark of everlasting light,\\
Above the howling senses' ebb and flow''?
\end{verse}
It is faith in the redeeming love of God. And
that conviction, as they will tell you, they
have and hold fast because they are sure that
Jesus Christ is Immanuel, that He came out
of the very being and bosom of God Himself,
and came at great cost. In Him they have
found the touch and breathing of the Father.
It is not manuals of theology which prove this.
Turn the pages of the hymn-book, and evidence
of where the Church feels the centre of gravity
in her creed to lie is discoverable on every side.
One instance must suffice. Matthew Arnold, you
remember, pronounced ``When I survey the
wondrous Cross'' the greatest hymn in the
\marginpar{96}
English language. It is at least one of the
greatest. Read the first lines of the second
verse---
\begin{verse}
``Forbid it, Lord, that I should boast,\\
Save in the death of Christ, \textit{my God}.''
\end{verse}
Here is the ever-recurring note. From the
very outset, faith has lived on the personal
presence of God in Christ.

If we have moved thus far, however, it seems
but reasonable to take a further step. I
mean that if we are trying for a view of God's
love which is really transcendent---something
than which we can imagine nothing greater,
because in subduing magnitude it goes beyond
all we could ask or think---we seem to gain
a standpoint where the idea of the preexistence
of Christ begins to count. It is an
idea which of course comes up repeatedly in
the New Testament---mostly by allusion, as
if too familiar to every Christian to need
comment or enforcing. For example, there is
St. Paul's glorious verse---all the more amazing
\marginpar{97}
that in his argument it forms a mere aside,
the careless riches (as it were) of the apostolic
mind: ``Ye know the grace of our Lord Jesus
Christ, that, though He was rich, yet for your
sakes He became poor, that ye through His
poverty might be rich.'' Or again: ``Have
this mind in you, which was also in Christ
Jesus: who, being in the form of God, counted
it not a prize to be on an equality with God,
but emptied Himself, taking the form of a servant, 
being made in the likeness of men; and
being found in fashion as a man, He humbled
Himself, becoming obedient unto death, yea,
the death of the cross.'' The Apostle is speaking 
of the coming of Christ to earth. He is
not dealing in metaphysics; he is dealing in
the deepest and purest religion; and, as Bishop
Gore has pointed out, what occupies his mind
is not the \textit{method} of the Incarnation, but
its \textit{motive}. He is totally absorbed and over
mastered by the vision of the grace through
which Christ had stooped down, so that in
consequence of this unexampled self-impoverishment 
we became rich, ``heirs of God,'' as he
\marginpar{98}
has elsewhere said, ``and joint heirs with
Christ.'' Now this means, if it means anything,
that the love embodied and conveyed in Christ
were so great for his heart and imagination
only because he thus caught sight of that
vast background of eternal being whose glory
must be sacrificed or laid aside ere Christ's
earthly career had its beginning. ``Any gift
Christ has for me,'' says a modern writer,
''depends on this, that He became poor. I
need a God to heal the trouble of my life,
but a God remote, inapprehensible, is no God
for the heart. He may have all fulness of
strength and wisdom and love, but if these
cannot display themselves they might as well
have no existence. Wisdom does not sit apart
from life, but proves itself to be wisdom by
entering into affairs and guiding them to worthy
issues. And love, also, is no abstraction;
it shows itself in loving, entering into conditions
which are foreign to it in order to prove its
quality. It takes upon itself burdens which
are not its own, it throws aside every privilege
and restriction, and plunges into the thick of
\marginpar{99}
common life. All that is in God could not be
known without an Incarnation.''

Faith, one feels, will always find it natural to
echo this conviction. Doubtless the conception
of Christ's pre-existence---His eternity were a
better name---is one of immense difficulty.
For here we light upon the enigma which in
religion confronts us at every turn---the relation
of eternity and time. It would scarcely be
going too far to say that every statement
of the doctrine of pre-existence which has ever
yet been made always contains those self-contradictions,
those manifest breaches of the
rules of logic, which indicate that the human
intellect is baffled. At the same time all will
confess that various points emerge in life where,
although we may not succeed in (as it were)
getting our hand round a truth, \textit{yet the truth
is there}. We can feel that the dimly-perceived
thought is essential for the interpretation of
experience; it is apprehended, even if not comprehended. 
No one has explained moral freedom 
convincingly, yet its reality is plain. No
one can tell how or why my will contracts the
\marginpar{100}
muscle of my arm, when I choose; yet the
contraction happens. In like manner one need
not feel debarred by the unquestionable difficulties 
of the idea from taking the pre-existence
of Christ as a supremely worthy symbol and
indication of an infinite, unnameable fact.
Whatever its defects, it is at least incalculably
truer than its negation. It wonderfully assists
the imagination when we are trying to form
a transcendent view of the Divine love which
gave Christ, that we should realize how His
being here at all meant sacrifice, sacrifice of
a kind and magnitude which ``pass understanding.''

Of course it may be said with truth and
point: God's love is visible elsewhere than
in Christ. Yes: it is visible elsewhere because
it exists elsewhere. It is there in the
\begin{verse}
\small
Relations dear, and all the charities\\
Of father, son, and brother.
\end{verse}
It is there in all high and saintly lives. Yet
in contemplating such lives we scarcely
feel ourselves in the presence of Divine self
\marginpar{101}
abnegation. We do not feel that they convey
Divine benefits purchased at a great price.
Such human exhibitions of love, purity, and
goodness are not such as manifestly to \textit{cost} God
much. But precisely this is what we feel in
Jesus. Standing in His presence, we are conscious 
of ``a love in God which we do not
earn, which we can never repay, but which
in our sins comes to meet us with mercy; a
love which becomes incarnate in the Lamb of
God bearing the sin of the world, and putting
it away by the sacrifice of Himself.'' \textit{That}
means stupendous renunciation on God's part,
for our sake. And this Divine acceptance of
pain, dependence, shame, and death, I repeat,
has gone to the world's heart. Ten thousand
times it has melted down in contrition and
gratitude those who must otherwise have remained 
stony, friendless, and despairing. In
common life we well know---ofttimes we know
it later with shame and unavailing sorrow---
the difference between sending a sympathetic
message to the suffering and going to their
relief in person. Of course the analogy is
\marginpar{102}
incomplete and must be drawn with care; yet
it does help us to form a worthy conception
of something great that was in God's mind
towards the world when Christ came into our
midst. It marks the significant distinction
between the idea of Jesus Christ on the one
hand as a prophet or messenger only, and on
the other hand as the Son who stooped down
to identify Himself with us in ``an act of
loving communion with our misery,'' that the
redeeming Life might be achieved under human
conditions. This is a difference which men
understand perfectly; and the tone of our
noblest religious conception---the conception
of God as Father---is altered, subtly but
unmistakably and pervasively, I believe, according 
as we choose the one reading of Jesus or
the other.

It is not---let this be reiterated once again---
it is not that God cannot be known as Love
apart from His Incarnation in Christ. It is
rather that; apart from Incarnation His love
is not exhibited so amazingly. It does not so
inspire and awe and overwhelm us. Great
\marginpar{103}
as the humanitarian Gospel is, we can imagine
one yet greater. When I read certain modern
books about Jesus---books of a noble spirit,
the authors of which one hails as brethren
in faith---thought seems to travel out far and
beyond all they have to offer, out to something
vaster, something still more subduing. They do
not make us feel that in Christ, His whole
being and doing, but especially in His cross,
God Himself is touching our lives and laying
hold of us. But faith asks for the very profoundest
meaning capable of being conveyed
by the words, ``God loves the world''; the
interpretation with most \textit{grace} in it, most of the
appeal that will reach and win the guilty.
And once again I suggest that when this thought
of Divine Incarnation goes out of Christianity
it makes a blank nothing else can fill. The
scale on which God's love is manifested is
changed, and can never again be quite the same.
This is no mere theological refinement but a
purely religious matter. A Christ who is
eternal, and a Christ of whom we do not know
whether He is eternal or not, are profoundly
\marginpar{104}
different objects, and the types of faith they
respectively evoke must differ widely in horizon
and in moral inspiration. If Christ grows on the
soil of human nature, as simply human, we shall
have to curtail our once glorious vision of the
self-sacrificing love of the Eternal.

It is the standpoint furnished by Incarnation,
then; which enables us to realize the quite in
expressible stake of religion in Christ's Person
as an embodiment, absolute and unsurpassable,
of the love of God. Let us recollect, as we
now conclude, that precisely this supreme
interest or motive went to shape the formulation
of what we call the doctrine of the Divine
Trinity. There operated the same desire to see
the love of God as constitutive of His very life.
When men began to clear up their minds as
to the implications of Jesus' self-chosen name
of ``Son''; when they inquired what it meant
for God to be Father in His inmost being,
it appeared to them that if God were Father
essentially, He must be so eternally and by
intrinsic nature. He did not begin to be
Father perfectly when the perfect Son was
\marginpar{105}
born into the world. He had been Father
from everlasting to everlasting, ``ere the worlds
began to be.'' But if God is Father eternally,
then Christ is eternally the Son of His love,
and the Father and the Son are ineffably one
in the eternal perfectness of the absolute personal 
life. The inner life of God, before all
worlds, we cannot think of save as the scene
of moral and spiritual relationships---of love
active, actual, and unimpeded; therefore,
since we regard Christ, and also the Spirit
given to testify of Him, as participant in the
supreme Godhead, we speak in faith of the
Divine Trinity. Thus we find a home for
Love in the depths of the Divine nature, ``not,''
it has been said, ``from any wanton intrusion
into mysteries, but under the necessity of
breaking silence.'' We see through a glass
darkly, but the realities we discern even thus
faintly are significant of the infinite richness 
and fulness of the life of God, Who
from the beginning has been sufficient unto
Himself.

Whatever may be said regarding this doctrine,
\marginpar{106}
whatever the uses to which it has been put---and 
not seldom they have been evil---the conception 
to which it points is at least a great
and impressive one. Nay more, at bottom
it is profoundly religious. We read again the
words, ``Father, glorify Thou Me with Thine
own self, with the glory which I had with
Thee before the world was''; and as their
solemn and elusive wonder lingers on the soul
we feel again how noble and subduing is that
vision of the One God which beholds Him as
never alone, but always the Father towards
Whom the Son has been ever looking in the
Spirit of eternal Love.

\subsection*{}
We have sought to contemplate the Lord
Jesus Christ reverently in the distinct aspects
of His being. We have dwelt on His attitude
to men in Palestine; we have beheld Him
as He still speaks and lives within human
souls; finally, and with a deep sense of
intellectual limitation, we have sought to
indicate His connexion with the inner life of
\marginpar{107}
God. What is the conclusion to which we
have been led? Was Jesus Christ a Teacher
of spiritual truth, who sealed His teaching
with a noble death? That certainly, but also
more. Was He the chief among the saints,
who still lives on in lives made better by His
timeless moral beauty? That certainly, but
also more. Was He the Word of God, the
transcendent message of the Creator to His
creatures, breaking the {\ae}onian silence of
Nature and revealing a Divine Heart in which
we mortals have an inalienable place? That
certainly; that beyond all doubt and question.
Yet when we look onward still, we find no
barrier to the veneration, the trust, the worship
with which He is to be regarded. ``He that
hath seen Me hath seen the Father.''

But let us not grow confused with many
words. It is in the light of a sinner's conscience, 
and only there, that the fact of Christ
becomes quite luminous. Within us all are
two great elemental impulses, two vital and
supreme desires. We crave an infinite \textit{gift}
which will satisfy even these insatiable hearts;
\marginpar{108}
a gift absolute, unending, eternal. We crave
an infinite \textit{object} also in which we may lose
ourselves for ever and for ever. At once to
take and to give in boundless measure; nothing
less will satisfy the heart. These two desires
are met in the Christ whom we have
studied. He is the Saviour, and He is the
Leader. His gifts to us are wonderful---sin
pardoned, sorrow lightened, death abolished,
heaven opened, and a present God in every
trouble. Through Him we are made personalities: 
no longer things, or links in a chain, but
free men. But also He is the Leader, imposing
on us an infinite demand. He leads us out
into ever wider pastures of truth and duty, of
service and self-denial from which there is no
discharge, in a bond of union with Himself
to which even death will make no difference.
All this Jesus Christ will be to you and
me. Rise up and claim this Jesus for your
own. Claim Him, not for self merely, but
for all who are dependent on your influence
for their thought of life. So abide in Him,
the Life and Light of men, that it shall be
\marginpar{109}
natural for you to turn to your neighbour and
your friend and say: ``I have found the
secret. I have found the Father. I have
found the Son of Man who is also the Son
of God.''


 
\end{document}