\documentclass[12pt,a5paper]{article}
\usepackage{geometry}
\usepackage{palatino}
\setlength{\emergencystretch}{3em}

% Mackintosh, Hugh Ross. “The Practice of the Spiritual Life.” The Expository Times 31 (1920): 312–16.


\title{The Practice of the Spiritual Life}
\author{Hugh Ross Mackintosh}
\date{1920}


\begin{document}

\maketitle

\textsc{There}\footnote{Mackintosh, Hugh Ross. ``The Practice of the Spiritual Life.'' \textit{The Expository Times} 31 (1920): 312--16.} 
is always a danger of supposing that some magic formula, some new crystal phrase, if only we could discover it, would solve all difficulties of the Christian life. Just as at the moment people are looking round for a panacea to cure the Church's ills, and suggesting that it may be found in better Biblical criticism or none at all, enthusiasm for social reform or quietistic renunciation of social interest, more ornate or more simple worship---so also it is with the individual. People wonder whether the remedy for mischiefs, personal or corporate, may not lie in some novel, mysterious idea; `if only,' as the old preacher said, `it would occur.' In point of fact, however, the sources of Christian goodness are known, and have been long open. They are as familiar and as great as the perennial themes of poetry---Nature, Love, the conflict of good and evil in human life. We Christians need not hunt about for the secret; it is an open secret. Our sufficiency is of God. Jesus said, `He that hath seen me hath seen the Father,' and in that said everything. Not more knowledge is wanted, but a better will. God is \textit{there} for us in Christ; the only question is, Shall we take Him in? The cure for our ills, social and personal, is just to be better Christians.

Again, we can have Spiritual Life if we long to have it. I recall an address by Dr. John R. Mott, in which the refrain came at intervals, like a strong hammer-stroke: `You can be holy if you wish to be holy.' Not that there is anything automatic in religion. But there is the promise of God to faith, and His promises get themselves fulfilled. 

In the Spiritual Life, we need the true \textit{inwardness} and the true \textit{outwardness}. There is reception, and there is expression. Probably most people have always been in agreement about reception, about the ways in which we are given the life of
\marginpar{313}
God. But they differ a good deal about expression, which is a serious thing. The ways in which we express the received Divine life undoubtedly react on the very presence and power of that life within us. Strip its leaves from a plant, and you may kill it; and give personal Christianity its wrong expression in the life we live alongside of others, or fail to give it the right one, and the consequences may be grave.

\subsection*{I.}

As to reception, the great believers by their experience have fixed one or two principles. They have marked down one or two sources of Spiritual Life as indispensable. Let us glance at these.

1. The Word of God.---There was an old saint who said that in former days he used to pray first, then read his Scripture portion, but he had learnt better and changed the order. He was right: Scripture, the vehicle of the gospel, must always come first; in it God takes the initiative, and our faith or prayer is a response. On the drill-ground the opening word always is, `Attention!' and the Bible calls us to order at the outset of devotion. `I will hear what God the Lord will speak' is the attitude faith takes to God, and it is in His Word, pre-eminently and unfailingly, that He does speak. If this reading of God's message is to be fruitful and serious, it must be daily. There was an advantage in the old days in being `masters of one book'; that strength we may recapture, for inquiry has made the Book more intelligible and more interesting than ever. 

What shall we read in the Bible? people say. First, Read what feeds your soul: which means that as you get older, new and before unappreciated portions of the Bible will disclose their value. Certain parts probably will come to no harm if you leave them alone altogether. But let first things be first: make the Psalms and the Gospels central. 

Second, How much ought one to read at a time? Where shall we stop? No man can make rules for his neighbour, but a counsel (not wholly original) may be ventured. Read on until you reach a verse where, if it be night-time, you can lay your head right down as on a pillow; or which, if it be morning, you can take in your hand as a staff to lean on for the day's march. 

Third, Occasionally read a book of the Bible right through at a sitting. When Dr. Moffatt's translation of the New Testament came in, I sat down and read `Philippians' from start to finish. How it freshened the whole to get the beautiful familiar letter in a new dress and in a single swift impression!  

We may take it that the Word of God is so essential that to speak of strong Spiritual Life apart from its constant use is folly. Some things experience does prove, this amongst them. Many new discoveries are being, and will be, made; but no one has yet found out how to nourish the body without food, and in the Bible is the spirit's food. 

2. Prayer.---If we breathe in God's redeeming truth by laying our heart open to His Word, with its nutrient properties, we breathe out the heart to Him in prayer. We speak in prayer, and we listen; listening is an element in prayer the importance of which we too much ignore. He who is never silent before God, listening in perfect stillness, cannot grow. Often the truth God tells us as we read the Bible, He repeats and seals as we pray. 

The inconceivable worth of prayer for Spiritual Life---this is not argument, it is Christian history. A reading of the great missionary lives is proof enough: Brainerd, Martyn, Livingstone, Coillard, we know whence they drew their power to set back the frontiers of darkness and let the light shine. Just as the arm of the electric street-car goes up and presses close against the live wire, and the car lies helpless and inert when contact is broken, so these men were weak apart from prayer. And to adduce the Name above every name, Jesus is our forerunner in this field. He is not the Saviour merely; just because He is Saviour He is also the great Believer. As we look at Him in the Gospels we can see that He was `the first that ever burst' into that great unexplored ocean of the Father's love and realized power to help. It was through prayer He got the good of that Love. He prayed by day and by night; with long petitions and with short; in the solitary mountains and in the crowded streets and lanes. Christ loved prayer and practised it. 

Nor must it be forgotten that prayer does men good only when they seek God for His own sake. All prayer that can be called prayer is uttered in the attitude of \textit{adoration}; the man who prays squinting at his own moral improvement defeats himself. That way lies self-consciousness. The object of prayer at its highest is not our success or felicity or holiness, but communion with God just
\marginpar{314}
for Himself. In the Lord's Prayer, God's glory and Kingdom take precedence of petitions for personal blessing. No man ever yet fell in love in order to improve his character, nor would his character gain that way; and if fellowship with God is to make us good, in the Bible sense of goodness, it must be because He is more to us than all His gifts. 

Mr. Oldham has said that when we think of prayer, we at once think of its limitations; when Jesus thinks of prayer, it is as crowded with unimaginable possibilities. There is nothing worth doing which it cannot do. Is not this specially true of `ejaculatory' prayer? No better saint has been in this country for long than Dr. Andrew Bonar, and in his journal he writes: `I find that unless I keep up short prayer throughout the day, at intervals, I lose the very spirit of prayer.' Nothing could be more \textit{natural} than such a habit; when staying with a friend, we do not speak to him at length before breakfast and after supper, carefully refraining from conversation in between. We remember that our friend is there, and we talk to him. There are many times, indeed, when nothing but sudden prayer will serve; moments of temptation, of perplexity, of the thrill of gratitude. Here too we have the pattern of our Lord. As He healed the deaf and dumb man, as He hung on the Cross dying in the dark, He prayed brief dart-like prayers. That should be enough for us. Let us not be like the child who said: `I didn't know you could say your prayers except of an evening.'

Do you pray? Even in this Convention one may safely put the question. Looking back over a week, can you see points at which you consciously placed yourself before God and took from Him the needed power? Were there moments at which I you laid hold of Him, and said something real, were it only `\textit{my God}'? Do not be put off by fear that you cannot pray for long periods. Probably you can't: very few people can. But we can take ourselves aside and see God's face. We can stretch our hands through the veil of sense and lay hold on the Unseen Love. We can have a little chapel, with an ever-burning light, where we kneel and receive.

% Poem
\begin{verse}
How would our souls stand up, O Lord, \\
\hspace{1em}Erect and strong and free, \\
If we but knew the ample hoard \\
\hspace{1em}Of wealth we have in Thee!\\
We do not need to sway Thy mood, \\
\hspace{1em}Nor beg of Thee to hear; \\
Ere our own mind has understood, \\
\hspace{1em}Expectant is Thine ear.
\end{verse}
  
3. Thinking about Christ.---Not that we are asked to think about Christ all the time; that is neither possible nor desirable. A student writing against time in an examination; a surgeon at a critical moment in an operation; a taxi-driver threading a crowded thoroughfare---their duty is to keep a mind concentrated on the task nearest them and not suffer their attention to wander for even a second, even to Christ. God knows this: it is He who has chosen these absorbing duties and sent us out to them. All the more reason we should use the leisure times that do occur to think about Jesus Christ in a natural and simple way. It can be done, for example, as we move along the street. An acquaintance once saw Dr. Chalmers, in Edinburgh, as he came down the Mound, his head sunk on his breast, deep in thought; watching him, he crossed the street and laid his hand upon his sleeve. And Chalmers looked up, like one coming out of a trance, saying: `That's a glorious verse---``My God shall supply all your need according to his riches in glory by Christ Jesus.''' Out of the heart's fulness the mouth spoke. 

How much easier the Christian life would prove, if only we thought of Jesus Christ oftener! If we had a dear friend in Australia, and never gave him a thought, he would even cease to be dear, and presently it would be all one as though he were dead. We are what we think about. The nation that is constantly dreaming of war, keeping its mind on the subject, goes war-mad; the man who keeps his mind on Jesus grows keen on all things for which Jesus stands. We abide in Christ by means of our thinking. Thought is the opening through which pour the waters of His great life, to flood the shallows of our poor nature.

To think of Christ is to enjoy His friendship, and can we set limits to what that friendship will do for us? It is an intimacy to enrich mind and heart. No one ever dreamt such dreams for mankind as Jesus, and we can listen as He speaks about them. No one ever so realized the supremacy of God's will, or so dwelt under its shadow; He can lead us also into that experience. No one knew so deeply that love means sacrifice; that lesson too He can instil into our narrow hearts. Will
\marginpar{315}
not this companionship, this effort, through His Spirit, to enter into His mind and taste its blessedness and delight---will it not make us different? Will it not bring us out of ourselves, therefore, from gloom into joy? Yes, it will. The indolent, the cold, the covetous---He can change them all. 

Clearly we make progress only as we look out---away from ourselves. Not self-inspection is the secret, but Faith. As Forbes Robinson put it in a wise word: `I have never found it profitable to meditate on my sins.' Looking up is so much better than looking in. That is why Faith makes a man stronger in character: it takes his mind off himself and fixes it on Another. So he ceases to brood I over failures or successes, and is changed by beholding. We escape from evil by thinking on what is good, and Christ is the best of all.

\subsection*{II.}

For a true full Spiritual Life reception must be accompanied by expression. 

1. Obedience.---Channels for the inflowing of the Divine life can be kept clear only by obedience. The Christian is a man who does as Christ bids him. `My Master has said such and such, and that is enough for me.' Have we ever taken this quite seriously---this duty to obey Christ? Probably each of us has some corner of life unreclaimed, unchristianized---our temper, our imagination, the way we make our income, our expenditure. We will \textit{not} let Christ rule over that. For one case of perplexity as what Christ's will is there are ten or a hundred cases of refusal to obey the will He has made quite plain.

Are you doing your best to keep His commandments? Remember this will react powerfully on your inner life: your fidelity to Him as Master affects your assurance that He is Saviour. To-day one of His commands is troubling many people. He bids us forgive our enemies. It is quite possible that God's hearing of prayer for Revival is going to depend on whether we are ready to forgive Germany. We know what Jesus said as He hung upon the Cross: `Forgive them, for they know not what they do.' Shall we contend that we have more unpardonable injuries to forgive than He?

2. Justice and love of our neighbour.---Our idea of saint is changing. The old mystic idea of \textit{solus cum solo} is not false, but if put forward as complete it is thoroughly unsound. A certain colour-blindness for definite parts of the Bible---such as the social teaching of the prophets and of Christ---has hid the fact that if we are to be saints, the people of God, we must rectify our relations to our neighbour. Note our Lord's answer to the question which was the greatest commandment. He began: `Thou shalt love the Lord'---which will always be primary and central and the fertile root of everything. But He did not stop there. He said there was a second like the first: `Thou shalt love thy neighbour as thyself.'

Therefore the saint must be a social reformer---in purpose, in sympathy---or he will not be a saint in Jesus' sense. Holiness means zeal for righteousness. If you are going to be Christlike in this sphere, which is it to be?---social reform an unpleasant necessity, lest worse should come, or social reform welcomed as the good will of God? We may find an analogy in slavery. If we discovered that an acquaintance of ours still held slaves---in Africa, let us say---we should be sure of one thing, that he was not a good man. Once a slave-holder \textit{might} be a Christian; we remember John Newton's statement that he had never had sweeter communion with God than on the deck of his slave-ship. Yet now as we look back, we say, `They were good men, they were in fellowship with God; \textit{but how could they do it}?' So when Christians look back a hundred years hence, on the Church of this generation, and mark the indifference to bad housing, sweated labour, intemperance, they too will say: `They were good people, they were in fellowship with God; \textit{but how could they do it}?' The idea of saint is changing, and it will change yet more. Mazzini, the Italian patriot, once said: `When I hear a man called good, I ask, ``Who then has he saved?''' Of more and more people within the Church it will be true that they have to catch this tide of concern for their neighbours' lives, or the great Divine movement will leave them high and dry. Their spiritual life will pay for their blindness to God's will. 

How am I to know whether I am making headway in the Spiritual Life? Here is a possible touchstone. Is Christ greater to me than ever? Is my sense of \textit{wonder} growing? Wonder at the love of God, wonder that we are His, wonder at God's passion in the Cross, at the infinite prospect of immortality? When Jacob Boehme lay dying,
\marginpar{316}
at the last he raised himself from the bed and cried, `Open the window, and let in more of that music!' That is where we want to live; with the music about us of God's unconquerable love in Christ. If something more of its marvel is taking possession of us, let us give thanks. `O Lord, I am thy servant, truly I am thy servant; thou hast loosed my bonds.'

\end{document}