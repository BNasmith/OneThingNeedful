\documentclass[12pt,a5paper]{article}
\usepackage{geometry}
\usepackage{palatino}
\setlength{\emergencystretch}{3em}

%% NB Still need to fiddle with quotation marks.

% Mackintosh, Hugh Ross. “The Revelation of God in Christ.” The Expository Times 27 (1916): 346--50.

\title{The Revelation of God in Christ}
\author{Hugh Ross Mackintosh}
\date{1916}


\begin{document}

\maketitle

\textsc{God}\footnote{Mackintosh, Hugh Ross. ``The Revelation of God in Christ.'' \textit{The Expository Times} 27 (1916): 346--50.} revealed in Jesus Christ---this idea is one which, judging by the quite natural difficulties felt upon the subject, requires no little explanation. All sorts of puzzles have accumulated round it. All kinds of objection have been raised to its validity, and the proofs led in its defence have occasionally been wrong-headed or irrelevant. 

To hold that in Christ we see God revealed is to
hold that if we Christians examine our own minds with regard to the content we ascribe to `God'---a term which, be it remembered, has borne and still bears a hundred different meanings---it transpires that we have carried over to God the moral attributes of Christ. God, in other words, is exactly like Jesus. No one really has ever believed that the world explains itself, or doubted that above or
\marginpar{347}
behind this phenomenal system of things and incidents there exists \textit{some} ultimate or supreme Power. The seen manifests the Unseen. But that Power has been described by many thinkers as Fate or Chance, by others as a mixture of Evil with Good, by some few as Unconscious Will. Christians refuse these descriptions, not arrogantly indeed or jauntily or coldly, but with intense conviction. They are persuaded that God is better than them all, because He is specifically the Power present in Christ. Our experience of Christ imprints this on our minds as self-attesting truth. Hence for us to think of God is to think of Christ with His essential characteristics exalted to infinitude. 

Now the revelation mediated by Christ is one calculated to meet and satisfy our religious needs. And it is constituted by this purpose. Jesus puts the Father within our reach, as faithfully and unchangeably Redeemer, and for those who have to live in a world of sin, transience and darkness, that means everything. The questions about God solved by the fact of Christ are questions not primarily of the intelligence but of the soul. That implies that various problems concerned with God are as inscrutable to the Christian as to the Shintoist; that faith, for example, gives no light on how God made the world or upholds it in being. These and other like matters are still opaque. What Jesus has done is so to unveil the Father that we have communion with Him. He enables us not to write essays about God, but dwell under His shadow. There is nothing made known in Christ which relieves men of the fag of thinking hard when they want to clarify their minds about all kinds of difficulty arising out of reflexion upon human life. The aim of revelation is a quite specific one. It is to rectify our personal relation to God by showing the Father in such a light as will bring us into fellowship with Him.

In that case, revelation cannot possibly be the same thing as the communication of theoretic statements about the Divine Being. Not even such statements with Divine authority to back them would avail, any more than to have read a man's autobiography entitles us to claim his friendship. There is nothing in doctrine as a purely objective affirmation of truth to guarantee God's personal interest in and love for me or to give me freedom of access to His heart. Nor is there anything in doctrine, still regarded in this light, which enables me to verify it in the daily life of faith. It is outside experience, with no chance thus far of getting inside. But if it should meet us in the living guise of a Person, if the truth should be embodied in a tale, the case is different. For this Person may be self-evidencing, and to look at Him with our nature laid open to His influence may change us through and through. That is what Christians testify has happened to them in Jesus' presence. Bowed before Him, they have come to know what God thinks of them and how He is even ready to suffer on their behalf. Through One whose influence over us is independent of time we find ourselves actually led to the Father. Even the word `revelation' may seem inadequate to the great truth. It is not that we place Jesus alongside of God, and bridge the distance between them by an inference. We do not argue from one to the other at all; we are made immediately aware that in this Man God is personally present.

People occasionally speak as if all this involved an absolutely new and unfamiliar principle. That, however, is not the case, and in fact an idea which was completely new would have no reality our minds could apprehend. It is worth pointing out, therefore, that as life proceeds we all of us, even apart from Christ, have revelations of higher truth, and that these revelations come through the spiritual impression made by persons. We believe, for example, in Friendship. We are sure there is such a thing as Friendship, that it has been manifested in indisputable ways, that it is the most precious thing in all the world. Why do we believe this? Because we have encountered those who exhibit its reality in their attitude and bring it in upon us as undeniable fact, undeniable at the very time when we seek to thrust it aside. That is a revelation, and persons are the medium. So too with Holiness. We are certain that Holiness is not a mere abstract noun, absurd and empty, but the most subduing and august of realities, because we have met and known those who are holy. The thing is quite easily distinguishable from all imitations, and when we come face to face with it we bow our heads in reverence and wonder. This also is a revelation, conveyed by the instrumentality of persons. Essentially in the same way, though on an infinite scale and with perfect efficacy, what Jesus is reveals the fact and the presence of God. He does not tell us about God merely; He draws us to behold Him, and by the sight we are changed. `He that hath seen me hath .seen the Father' is
\marginpar{348}
the plainest transcript of life. It is not something we are to believe because Jesus said it; it is what our experience of Jesus means. 

What, then, are the main features in the impression of God we receive from Christ? Let us take Christ at one particular point in His career and do with it as men do with a noble picture---stand before it, and let its meaning sink into our mind. Let us select His attitude to the woman that was a sinner. It is instinct, for one thing, with that Love to which we give the high name of Grace. She was an outcast, but Jesus went much further than to touch her; He suffered her to touch Him. The delicacy of His feeling, His kindness, His longing to uplift and console and heal, His sympathy with the fallen one, His trust in her repentance---this was wholly unprecedented in her life, and it made all things new. She felt, without reasoning, that in Jesus she was meeting that than which nothing can be higher, and that when He said `Her sins, which are many, are forgiven,' it was the voice of God. But observe, this Love, so gentle with the sinner, is none the less implacable to sin. It is a holy Love. Stained men and women, now as in the first century, are confounded and humbled by that stainlessness, which not only evokes a sense of ill-desert, but imparts both depth and passion to their penitence. A man cannot take down the Gospels and use half an hour in reading three chapters of Jesus' life without arriving at certain absolute conclusions, and of these one is that God is holy. It is from Jesus we gain that certainty. His eyes look out upon us from the page, and through them shines, inescapably, the holiness of God.

Antecedently we might suppose that Love and Holiness are incompatible in their supreme form. In our acquaintance sympathy and righteousness do not always go together. Holiness, men have often believed, is the attribute which puts sinners at a distance and keeps them there. And certainly none were so sure as the guilty who approached Jesus that He could make no terms with sin in a disciple's life. And yet, having sought them out, He stayed on beside them with a personal concern which was at once appeal and promise. So that it is only when Love and Holiness fall short of perfectness that they move apart and issue in antagonism; then Love becomes weak, and Holiness grows coldly exclusive. But in God, in Jesus who is the image of God, they are as inseparably one as the concave and convex aspects of a curve, and the Holiness by which we are abased is one thing with the Love that lifts us up and makes our moral being rejoice. 

To see Jesus, therefore, is to become aware of that Holiness and Love \textit{in excelsis} which for Christians are the equivalent of the moral nature of God. He is their presentation in history. But, as we believe, God is more than Holy Love, He is Holy Love which is \textit{almighty}. Can this further element be derived from Jesus Christ, that is, from the immediate impression left upon us by the Gospel picture of His life? The difficulty at this point is greater.

In one sense, indeed, there is no difficulty at all. It is clear that Jesus conceived the Father as omnipotent, and in this respect shared the highest faith of the Old Testament. It is little indeed to say that He shared that faith. To quote a recent writer: `One cannot make an unprejudiced examination of the Gospels without being astonished to find how enormously important for Jesus' view of God was His impression of God's omnipotence and infinite sublimity. I am very far from failing to recognize that in His apprehension of God Fatherly love constituted the central feature. But the importance of this extraordinary fact can be rightly appreciated only so long as one realizes that His view of God did not emphasize the Divine power, majesty, and sublimity one whit less than did the Jewish view, but took the latter for granted---nay more, deepened it and intensified it to the absolute uttermost.' But assuming this, we have still to ask whether the sense of God as almighty which Jesus gives is differently conditioned from our sense of His holiness and love, in this respect that whereas the holiness and love of God come home to us directly in Jesus' presence, as intuitively apprehended in our very apprehension of what Jesus is and does, the truth of Divine omnipotence is mediated only through what Jesus believed and said. In that case, our faith simply rests on \textit{His} faith. But I think that we are really able to go further. For one thing, power is itself a manifest element in Jesus' work. Though we leave aside nature miracles as disputable (many of His healing works, in point of fact, are as wonderful as any nature miracle), the redeeming energies He brought to bear on men in performing upon them the comprehensive miracle of salvation do indicate such power as only needs to be raised to the absolute
\marginpar{349}
scale to represent Divine omnipotence. Men upon whom Jesus laid His saving hand became aware that there was power in Him, as well as holiness and love, which spontaneously led their minds to God and gave them a quite definite conception of what God can do. And if it be replied to this that such a direct impression of power concerns the spiritual realm merely, and is irrelevant to the physical universe, our answer is that faith rightly declines to separate the two. The universe is one, and if it is such as to admit of a Person almighty in His character as Saviour from sin and sorrow, those spiritual energies in Him which reveal God must in God be accompanied by unlimited powers also over nature. The thought is not so much logical inference as rather a movement of believing intuition. The redeeming might of God presented in Jesus is master of natural law.

Now this compound yet simple conception of Almighty and Holy Love is precisely what we Christians mean by God. Its constitutive significance is all present in Jesus, and is nowhere else fully present in history or nature. This is the meaning of our immemorial belief that God is revealed, and, for the purposes of religious faith, perfectly revealed in the historic Christ. The knowledge of God indispensable for a life of peace and joy cannot be gained by hard thinking, or by scientific inquiry, or by the scrutiny of our own constitution; it can be gained only by laying bare our moral nature to the impression left by Jesus in the Gospels. We find in God nothing else than Christ. 

To this revelation there belong certain conditions or attributes which it is worth while to set out in distinct terms. 

(\textit{a}) At every point it is mediated through ethical experience. It comes to us through the living and breathing substance of free and unselfish motive, not invading personality, not forcing or outraging conscience, but winning us by being what it is and shining solely in its own light. Had revelation consisted in the imposition of divers theorems concerning God, belief in which was prescribed as the gate of entrance for all, religion would have unequivocally defined itself as the foe of morality, for such an externally authorized creed, depressing by its very mysteriousness, would have added to our load, not lessened it. But what we see in Christ imposes its truth upon us freely; it is echoed by the voice of conscience; it evokes just such a belief and loyal confidence as a man has in his friend. The human spirit is never so much at liberty as in the moment of joyful response to Christ's presentation of the Father. 

(\textit{b}) Revelation is supernatural in quality and range. By this I mean that it is something which no phenomenal realities of our normal world can explain in the very least. The communion of God with men through Jesus is miraculous in the sense that nothing explains it save the intervention of the living God in a sense not to be accounted for by the resident forces of human life, or the intramundane causal nexus. God acts freely in unbaring to us His heart; He releases into the phenomenal order the stored-up energies of His grace; and this is borne in upon us convincingly by the fact of Christ. But revelation is not miraculous in the sense that it discards finite media. Jesus, too, is part of the world. When God poured the fulness of His being in Christ, it was a living intrusion in the human sphere, in ways not derivable from known laws or the given phases of the universe.
 
(\textit{c}) The vehicle of revelation is history. To-day fewer men than ever profess to find a saviour God in Nature, but there are still those who would call or recall us to the ideals of Reason, with the promise of perfect satisfaction. But the impotence of ideals to produce their own actualization is the theme of moralists ancient and modern, great and small. Can any one feel the value of sobriety like the drunkard; is any less inclined to deny the loftiness or the necessity of self-control than he? Now one truth humanity is slowly learning---Christians have known it from the first---is that history is immeasurably richer in impulse and contribution than any single life. The victorious \textit{differentia} of our religion is that it is no system of ideas or ideals suggested by the Spirit to the souls of men, but a story of definite acts done by God before men's eyes. Redemption is mediated through One who belonged to our own sphere of reality, who trod the earth our feet are treading now, who lay down in the grave and on Easter morning broke the power of death. Christianity has the life-blood of fact in its veins. The preacher is able to stand up and preach not what he feels, but what God has done.

(\textit{d}) Revelation is an appeal to faith. In other I words, it speaks with a resounding voice, but only faith can hear. There is no automatic action
\marginpar{350}
of the Divine self unveiling on the soul, and if people want to shut their eyes to God's presence in Christ, they can shut them. The impression is made solely on the right kind of mind, the mind that hopes there is a God, and hopes, too, that He will lift the veil and betray His purpose. That is a principle not in the least confined to the religious life. It holds good equally of art. The meaning of a great picture or a great symphony is not the creation of the susceptible spirit to which it is presented, but without susceptibility of spirit, without the right kind of mind, no impression at all will be made. 

Theology has not greatly inclined to deny the fact that God is revealed in Christ, what it has often done is to cancel this truth either by taking its point of departure elsewhere than in Christ, or by admitting as equally valuable sources of revelation other fields of experience which belong to a lower ethical plane, such as Nature or the general history of the world. It cannot be too strongly asserted that a Christian's only legitimate method is to make Christ the starting-point, thus ensuring that His influence shall fix once for all the main outlines of our thought of God. Anything else is to court disaster. Moreover, the revelation of God in Christ has no need to be improved upon. Had improvement been called for, we may well believe it would not have been withheld; but in point of fact no vital element has ever been added to the conception of the Father as imaged in Christ. What has happened is a vastly extended application of principles first embodied in His person. It is still as true as in the first century that Jesus `reflects God's bright glory and is stamped with God's own character' (He 1\textsuperscript{3}). Nothing can be allowed to interfere with this---not science, or philosophy, or non-Christian religions. Christ is the revelation of God our Father---final, unsurpassable, and, in a sense which faith quite well understands, absolute. All that we have to say (and it is much) about the unveiling of God in the Old Testament, in the course of history and the constitution of man, or in the world of Nature, must be subsumed under, and controlled by, the self-delineation He has given in our Lord.



\end{document}