\documentclass[12pt]{article}


\begin{document}


\textsc{My} first task this evening is to remove a misapprehension which may be
present in some minds. By standing here, and undertaking to speak upon
this subject, ``How is God Known?'' I am not of course professing to
give you some short and easy programme, by following which, if a
stranger to God up till now, you might reckon on coming to know Him
without fail. That would be an irreligious proposition. We dare not, and
in any case we cannot, prescribe to God the ways in which He shall act
in making Himself known to a man. All we are sure of in advance is is
that whatever path you may have to follow to reach His light, you will
be certain after you have come to know Him that it was the best path for
you.

\paragraph{I.}\label{4039}

Probably I can take it for granted that many of us feel that the
question ``How is God known?'' is a painfully difficult one to answer.
There are, of course, people to whom the knowledge of God does come
easily. Circumstances, and the make-up of their own inner life, are such
that they can hardly remember a time when they did not know and love
God; their faith unfolds quietly, imperceptibly, as a flower spreads
itself beneath the sun. Well, why shouldn't it, if their life has lain
that way? They are enviable in this respect, anyhow, that by their clear
faith they have been helping other people for years before the rest of
us got into our stride. But for a good many others it is utterly
different. It is the difficulties that loom up before them. What strikes
them most intensely is the chaos of the world; its apparent
meaninglessness, drifting like a river in space. How can anyone expect
to see God through the poisonous fumes of class antagonism, or the fog
of international hatred and racial folly? Can I easily believe in God's
love if He has given me a nature that tortures me by its fierce
ingrained appetites? You see, these people have discovered the great
fact that God is \emph{hidden} by the world, as well as revealed. Never
forget that much in the world is utterly alien to God's
nature --- cruelty and cancer for instance. His will is opposed, dead
opposed to these facts; so that to speak of them as phenomena through
which He is transparently manifested is absurd. We must never identify
His will just with what happens. These evil things --- and there are
mysteriously many of them, in the world and in
ourselves --- \emph{obscure} God, and if we come to know Him, it is in
spite of them, not because of them. You can't rise from cancer up to
cancer's God. The New Testament is far more open-eyed and plain-spoken
about these facts than we are. It is clear, indeed, that the apostles
\emph{only} \emph{just} overcame these difficulties; they triumphed over
the stark actualities of tragedy and death by a faith which \emph{all
but} broke under their weight. Something wonderful had happened,
something connected with Jesus, which actually enabled them to rest in
the knowledge of God in spite of these dark things.

\paragraph{II.}\label{813d}

Let me start with this point, that our knowing God is not the primary
thing; rather it is the result of something else. You will search the
world in vain for anyone now living in personal fellowship with God who
believes that he got there simply by self-inspired exploration. The
saints never claim to have done it themselves --- taking up religion, as
previously they had taken up golf or politics. Our explorations are
often such noisy affairs that they drown the finer intimations that are
trying to get through. Getting to know God is fundamentally a matter of
\emph{listening}. He takes the first step in this business and looks to
us to respond. We all know the kind of man who pretends to be consulting
us, but talks all the time himself. He pours out his own ideas so that
it is impossible to get a word in. Now the Christian case is that God
has spoken and is speaking; are we willing that He should have a chance
to make Himself heard? Our knowing God is a response to His voice. In
our everyday knowledge of the surrounding world it is not the fact that
by taking notice we create the objects we perceive; the mountains, the
moors, the sea --- they were all there in their solemn grandeur and
loveliness long before we were good enough to pay them attention. And in
a far deeper sense everyone to whom God has become the One supreme
insistent Reality is ready to confess that God was there --- active,
calling, pressing in upon the soul --- before we wakened up to His
revealing presence. We know God, if we do know Him, because He knew us
first. He sought us before we found Him, and it was His seeking that led
to our finding. The beginning of everything is that God addresses men,
places Himself in their path, stops them, and challenges them. Religion
is a bestowal, not an achievement. If we know God, it is because we have
let Him speak to us.

One thing, I imagine, it is possible to assume, namely, that knowing God
will not be an experience of the same kind as knowing the multiplication
table, or, say, the chemical elements and their properties. Let us clear
up our minds here; it is lamentable how many people go about the world
assuming the opposite, whether insisting upon it or deploring it. By far
the great half of the knowledge we possess is not scientific in the
least, and all the better for that. When the physicist is at work in his
laboratory, he practices one kind of thinking --- scientific dissection
and construction you may call it for short. Then he goes home to his
family and friends; but he does not bring scientific dissection to bear
upon \emph{them}; there he goes in for a quite different sort of
thinking; he knows them by intuition, trust, sympathy, love. Professor
Eddington has a delightful passage on the subject. ``The materialist,''
he says, ``who is convinced that all phenomena arise from electrons and
quanta and the like, controlled by mathematical formulæ, must presumably
hold the belief that his wife is a rather elaborate differential
equation; but he is probably tactful enough not to obtrude this opinion
in domestic life.'' I suggest to you that if there be a God after the
pattern of Christ, then knowing Him will not be a scientific
affair --- anything of the kind would be quite irrelevant, as irrelevant
as the rules of chess to a game of football; it will rather be like the
way we know in personal relationships. And that is as real and
trustworthy a type of knowing, to put it mildly, as any other.

\paragraph{III.}\label{28ff}

Now that at once suggests a point worth looking at. I have been
stressing a real analogy, helpful for our purposes at the moment,
between knowing God and knowing an acquaintance, or better, a friend.
The two are similar in this respect, at all events, that utterly
insoluble puzzles emerge in both cases. For example, does your friend
have a real personal identity? If you say: Of course he has, could you
prove it in any way that would silence the objector --- Mr Bertrand
Russell, for example --- once and for all? I wonder. Is he your friend
because you trusted him, or do you trust him because he is your friend?
Which came first, the trust or the friendship? Or again, do you
understand how that friendship arose? Isn't there a kind of mystery and
wonderful surprise about its coming? When the gates of that new world
opened, could you tell exactly how the thing came about? If you analyse
it in retrospect, isn't there a point at which, do what you will, the
thread of explanation breaks, and all you can say is that at a certain
stage you felt in your bones that this new, creative, inexplicable
experience had been \emph{given} you? All I can say is, that's how
things look to me; and, if there were time, a good many poets, who had a
right to speak, could be called in evidence. And yet, with all these
puzzles and difficulties, we \emph{do} know our friends, and can trust
them not to let us down. Now knowing God is a certain kind of
friendship; it is friendship with a difference, since He is God and not
man. But it \emph{is} a kind of friendship. Unless something like this
holds true, it is impossible to understand why Jesus Christ ever lived
in this world at all; why He looked into people's eyes, and spoke to
them, and took a grip of their hand, and stood by them to the very last.
He did these things to give men God's friendship, something to hold to
in spite of the painful puzzles of the world.

Follow the analogy another step. You gained that friend because he
disclosed or revealed himself, and that intense revelation came through
his word. He said things which let you see his personal attitude to you,
or he did things which had a significance that could be put in words;
and through just such manifestations of his underlying mind and
intentions, the friendship, the fellowship of spirit with spirit, began.
You could not have come to know him in any sense that mattered if he had
wrapped himself up in a cold taciturnity, and utterly refused to give
himself away; no, and you could not have come to know him, either, if
you had met his advances with a suspicion and distrust which
misinterpreted the simplest actions. If there was to be friendship at
all, there had to be speaking on the one side, and listening on the
other. It takes two to make a quarrel, we say; and similarly, it takes
two to make friendship.

\paragraph{IV.}\label{d754}

The whole question then may summarily be put in this way: Has God
spoken, calling to us for our faith and obedience and friendship? Quite
possibly, if He has, some people none the less may not have been able to
make out His voice; but that in no way demonstrates that the others, who
\emph{have} heard something, something that changed their lives, were
victims of hallucination.

For example, He has been heard to speak in the beauty and grandeur of
Nature. I wonder whether the passages I am going to read will strike you
as somewhat high-strung or sentimental; to me it seems an exact
description of what has often happened. Even if you turn it down, yet
you will enjoy it as a piece of English prose. ``As men watch the
appearance of the sunset,'' says Illingworth, ``thoughts and feelings
arise in their hearts that move their being in unnumbered ways. Youth is
fired with high ideals; age consoled with peaceful hopes; saints as they
pray see heaven opened; sinners feel conscience deeply stirred. Mourners
are comforted; weary ones rested; artists inspired; lovers united;
worldlings purified and softened as they gaze. In a short half-hour all
is over; the mechanical process has come to an end; the gold has melted
into grey. But countless souls, meanwhile, have been soothed, and
solaced, and uplifted by that evening benediction from the far-off
sky.'' You might say that such intimations of Nature are vague,
ambiguous, precarious; you may say they are decipherable only by those
who already had come to know God otherwise; and in a large degree I
should admit it all. Yet no amount of qualification can destroy the
fact, I think, that in every age great souls have known God, up to a
point, through the voice that speaks in Nature.

Again, most of us have felt the appeal of human lives higher and better
than our own. We feel that behind such lives there is a Power that is
more than themselves. Such people don't usually wear their heart on
their sleeve; but occasionally, in critical or tragic moments, the
secret breaks out, and you find that they are living by faith in the
Unseen. Now my own opinion, pretty emphatically, is that everybody who
knows God has come to know Him by watching, or feeling and submitting to
the impact of, just such lives. We discern --- suddenly or
gradually --- the inner meaning of their power to evoke our reverence;
we discover that the explanation of what they are is God; God's reality,
and His character, dawn upon us through their unconscious influence. Is
there a single person here who has never encountered lives like that?
And once we wake up to their goodness, which shames our selfish evil, do
we not begin to understand that in their goodness we are face to face
with something far higher than merely human acquisitions of virtue? I
say without hesitation that through such men and women, whom we have no
option but to reverence, God Himself is addressing us personally, and we
have positively to stop our ears if we are to make what He says
inaudible.

And yet even all that is not sufficient. The best people we know break
down somewhere, and they are themselves the readiest of all to confess
it. We therefore turn to One who never breaks down, never disappoints
us, One whom we do not praise because He is above all praise. I don't
expect there would be dissent from any quarter were I to say that if God
can ever be known aright and satisfyingly, it is through Jesus Christ.
To those searching for the kind of God it would be worth while believing
in, the choice is between Christ and nothing at all. Even if men protest
that the idea of a Loving and Righteous Father is only a dream, however
lofty, still it is from Jesus they have gathered the contents of their
dream. When we think of God, it is the face of Christ that rises before
us. And the fundamental Christian faith is this: that a world with Jesus
in it has a Loving and Just God above it. From the first century till
this hour it has been the conviction of innumerable hearts that Jesus
did not go too far when He said: ``He that hath seen Me hath seen the
Father.'' They have come to know God by getting to know Christ. How they
have done that --- I don't suppose you can explain perfectly and without
remainder, any more than you can the rise and progress of fellowship
between your friend and yourself. But the whole of the spiritual
experience of the saints is behind me when I say that if a man will stay
in Jesus' company, familiarizing himself with the Gospel portrait, not
too proud to learn, humble enough to pray, ready to do God's will, then
he will see through Christ transparently on to his ultimate object, and
his mind will open to the Father as \emph{his} God. God is looking into
our eyes through the eyes of Jesus. In Jesus, God is personally present,
offering His infinite, unchanging friendship to every man, woman and
child in the world.

\paragraph{V.}\label{bb5e}

Now the man who is resolute and serious enough to face Jesus, and let
Jesus tell him the truth about himself and his complete moral failure,
is inevitably called to \emph{decision}. Without decision there can be
no real knowledge of God. We know God when we made up our mind for Him;
and anything else is only playing at religion. He comes in upon us, and
corners us, so that we have to say Yes or No. You can't treat this
matter in a spirit of genial detachment; there is no other question in
the same category as the question of knowing God; here we are dealing
with a question of life or death. The attitude of a disinterested
observer is an insult. The mere observer is uncommitted, and therefore
\emph{blind to the issues}. No one ever knew God without taking sides.
Christians know that the revelation of God in Christ is true, because in
revealing what God is, Christ also reveals what we are, in our sheer
failure, and calls us to choose between God and self. No other in all
history compels us to face that tremendous and inescapable alternative,
and it is for that reason that no other in all history unveils the face
of God and makes Him our personal possession.

Another way of putting the same truth is this. There is no blinking the
fact that we can only know God if we feel we are altogether unworthy to
know Him. We cannot stroll up to the question complacently, in a spare
half-hour, and dispose of it coolly and dispassionately. One certainty
we must take with us in the search, if it is to lead anywhere, is that
only by getting to know God shall we ever be any good. Sometimes we are
told that we shall find God by turning to the best that is in us; but,
in the last resort, that affords very little help, for surely anyone who
possesses even the faintest sense of humour, not to say a sensitive
conscience, knows perfectly well that if there is a knowable God, He is
infinitely, unspeakably better than even the best in ourselves. God can
speak to us, of course, \emph{through} the best that is in
us --- through our sense of right and wrong, for example --- but that is
another story. But we cannot identify Him with any aspect of our being,
even the best; He is definitely other than we are; and just for that
reason He can be our Saviour. Why, what we need supremely is to escape
from ourselves --- even our best --- into quite another world of
perfection. There is a passionate humility at the heart of any knowledge
of God that counts. I must know God, for neither myself nor my friends
can deliver me from my own past. I must know, or I shall die.

\paragraph{VI.}\label{21c4}

To know God, and live in fellowship with Him, is the one secret of being
in fellowship with our neighbour. We cannot have God without having the
others too. This is a matter of our whole life --- inward and outward,
personal and social. To settle this question is by implication and in
principle to settle everything, though the application of the principle
in details may often be desperately hard. But then that is just what the
Christian life is for. To be a Christian means the admission that we are
only beginning to know God, that we have only started to explore the
meaning of Christ, and that we have the infinitely interesting prospect
before us of continually learning more about Him, through the joys and
sorrows and tasks appointed us.

Can we find God here? He has undoubtedly been found at other
Conferences; why not again? But whether men and women are going to begin
to know God before these meetings are over does not really depend, we
are all aware, on any arguments I have offered you to-night; it is a
question to be fixed and solved between God and themselves. Do you want
to meet Him? Would you be relived if nothing happened, or have you got
it quite clear that finding Him is the supreme necessity? The one quite
certain thing is that He longs to have you know Him, and that in such
knowledge there is eternal life.

\end{document}
