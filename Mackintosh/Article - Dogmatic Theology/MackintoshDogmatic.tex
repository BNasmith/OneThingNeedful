\documentclass[12pt,a5paper]{article}
\usepackage{geometry}
\usepackage{palatino}
\setlength{\emergencystretch}{3em}

%Mackintosh, Hugh Ross. ``Dogmatic Theology: Its Nature and Function.'' The Expositor, sixth series, 10, no. 6 (1904): 413--31.


% Found at: https://biblicalstudies.org.uk/articles_expositor-series-6.php



\title{Dogmatic Theology: Its Nature and Function}
\author{Hugh Ross Mackintosh}
\date{1904}


\begin{document}

\maketitle

\footnote{ % Original footnote is after the title.
Inaugural Lecture to the Class of Systematic Theology, New College,
Edinburgh, Oct. 20, 1904.
} 
\textsc{The}\footnote{
Mackintosh, Hugh Ross. ``Dogmatic Theology: Its Nature and Function.'' The Expositor, sixth series, 10, no. 6 (1904): 413--31.
}
unambitious task we shall essay this morning is to
gain a precise and comprehensive notion of the subject on
which the class of Dogmatic Theology is to be engaged.
In one sense, of course, this is impossible at the outset. A
great philosopher has said that clear self-consciousness is
the last result of action. It is not the man who is in the
middle of doing a thing that knows the meaning of what he
is doing, but the man who has come to the end, and looks
back. The dictum is as true of sciences or of theories as
of the history of a nation or a Church. At the outset, that
is, you cannot condense a treatise into a phrase and call it
a definition, nor would it, if you could, be of the least use
to those who wish to begin at the beginning, and to
form their conceptions of the science in question as they
proceed. At the same time something of practical value
may be done in the way of description, if not of definition
strictly understood; and obviously to secure a working idea
of the subject-matter of Dogmatics as well as of the methods
proper for its treatment, whether it yields much positive
enlightenment or not, may prove of considerable benefit in
preserving us from erroneous prepossessions.

To diverge into history for the moment, it is a not unimportant
detail that Systematic Theology, in its older
signification, embraced what we are accustomed to regard
as three distinct theological disciplines---distinct, that
is, in treatment, but not really separate in fact. These
were Apologetics, Dogmatics, and Ethics; the last having
commonly attached to it the epithet ``Christian'' or
''Theological,'' to mark the difference that obtains between
it and the more general science of Philosophical Ethics, or
\marginpar{414}
the Theory of Morals. Men were quite aware, of course,
that these three members of the organism of Systematic
are in the closest possible connexion. They all deal with
Christianity as a definite truth or power at work in human
life, viewed in each case, however, from a slightly different
angle. You may say, for example, as K\''{a}hler does, that all
three are concerned with justifying faith in Christ; Apologetics
taking as its province the grounds of faith, Dogmatics
its import, and Ethics its practical expression in life. Or
in simpler English, you may say that they deal respectively
with the presuppositions, the content, and the practical
issues of saving faith. In either case, their common interest
in understanding what the Christian faith is and implies
signalizes the truth, not only that they are distinct, but
that they are \textit{related}, which here and everywhere else is
very far the more important fact of the two. It is, then,
with the second of these kindred, and, in a sense,
co-ordinate studies, that we in this class are concerned.

Now in the general title of this department of theology,
there occurs a very significant adjective---Systematic. That
tells and foretells not a little. It means that we are at
work upon a subject which is a whole---not a collection of
alien fragments of knowledge, not a combination of interesting
but inconsecutive ideas, but a whole. If Christianity
is really one---and, this is certain if Christ is one---theology,
which is a sustained attempt to exhibit Christianity
to a believing mind which is also a knowing mind, 
is a unity too. Every part of it is in vital connexion with
every other part. We speak of the Caledonian Railway
system, and by that phrase we mean that we can get from
any one point to any other without hiatus or break. In the
same way, if theology be the outcome of an effort to present
the doctrine of God implied in Christianity, it will share
the unity which Christianity itself has, in virtue of its
source and object; and the links between its different
\marginpar{415}
portions will be continuous. And thus when we theologise,
or define the knowledge we have of God through Christ, in
order to translate it into scientific form, all our labour rests
upon the presupposition that our faith is genuinely a
whole, and can be shown to be so. We take for granted
that at no point shall we be put to confusion as intelligent
or religious men, by coming upon a doctrine which is isolated
and incoherent, wholly out of relation, ``like a rock in the
sky.'' And this assumption, as we cannot too often
recollect, is itself a warning and a test. For if we find in
the traditional theology this or that element which is in
imperfect relation to the Evangelical conception of Jesus
Christ, the centre and core of the entire doctrinal construction,
there is, as we instinctively feel, something wrong
somewhere. Either tradition has turned down a wrong
road at this point, and failed to approach the truth in
question by the avenue proper to Christian thought; or the
doctrine itself is a mere excrescence, an incubus because a
superfluity, and must be straightway cast out. There is
room in the Christian system for nothing but saving truth.

It may perhaps be objected that the claims now made
for the quality of system in Dogmatic are excessive, especially
in view of the clear statement of St. Paul that ``we
know in part, and we prophesy in part.'' It is indeed true
that theology must ever be only in part. We know no
more than sinners deserve to know, and that is but a
fragment. And besides that, a tentative and incomplete
character is forced on theology by the inevitable circumstance
that in dealing with religion, it is dealing with a
living thing. Life, by its very idea, is the perpetual despair
of thought. Experience, when we begin to reflect upon it,
is already something that has been lived through; and in the
very act of coming to full self-consciousness, it has parted
with a certain element in its freshness and its passion.
Meditation comes halting in the rear of personal history,
\marginpar{416} 
and while we are analysing what we thought and felt
an hour ago, some further thing is possessing our heart
already. ``When philosophy,'' said Hegel, with a touch
of melancholy, ``when philosophy paints its grey in grey,
some one shape of life has meanwhile grown old: and grey
in grey, though it brings it into knowledge, cannot make it
young again. The owl of Minerva does not start upon its
flight until the evening twilight has begun to fall.'' And
if theology, in this respect at least, shares the fortunes and
partakes in the deficiencies of philosophy, is it not a mistake,
it may be said, to claim for it the high and august character
of a system? Is not this as much as to say that, as an
explanation of things, or at least of Christian experience, it
is adequate and final? And how is this to be combined with
the certainty, of which the believing mind cannot divest
itself, that in the Christian salvation there is a vast residuum
of as yet unappropriated truth, an unfathomable deep of
gracious meaning out of which new and unforeseen disclosures
may at any moment, and do from time to time,
break forth?

Considerations such as these are deeply impressive; they
are so true, so peremptory, so undeniable. Yet we may
surely concede their truth without prejudice, as lawyers
say, to the idea of system, of proportion, of organic and
reciprocal interdependence, which we take to be characteristic
of the diverse elements in the theological structure. The
quality of wholeness is implicitly present in religious belief,
because it is present first of all in the reality which belief
apprehends. No doubt there are degrees of knowledge;
yet all degrees are animated and explained by the ideal of
an articulate unity to be known. The forester, the botanist,
the painter study the tree before them each with a different
interest; nor do the conclusions of all three, when summed
together, exhaust the meaning of the tree for a perfect knowledge;
yet it is only as a living whole that it has any reality
\marginpar{417}
for their minds. On every hand we are confronted with
unities which are indivisible because they are alive, and their
members, though logically separable, interpenetrate each
other, and are always more or less united in existence and
operation; we know them as wholes, even when we fail to
discover what it is that makes them wholes. And for us,
in our study of Christian truth, the same assumption is
indispensable; while, as to the grounds on which it may be
justified, provisionally we may say, as has been said already,
that Christianity has its unity in and through Jesus Christ,
the consistency of His Person, the coherent oneness of His
work and influence. Christ is not divided; therefore the
divisions and subdivisions of our systems are less final
than they seem. The doctrines have a right to live only as
they hold their life in fee from Him, and bring some real
aspect of the eternal grace that is in Him to expression.
Without this conviction the theologian cannot start; he does
not feel it worth while to go on. And above all, for those
who are to preach Christianity, it must be a point of settled
conviction that the contents of our religion form a single
organism of truth, capable of consistent and unified statement,
and that the secret of this unity is Jesus Christ.

Already I have had occasion more than once to use the
word \textit{scientific}---as when I said that in theology the attempt
is made to put our knowledge of God into scientific form.
But what is meant by the term scientific when employed
in this connexion? To answer this natural and indeed
inevitable query, it is needful to remember that the word
science may be used of the study either of \textit{things} or of
\textit{persons}; and that its connotation is bound to vary according
to the objects upon which it is directed. In physics,
chemistry, botany, the mind is dealing with \textit{things}, and
dealing with them scientifically; and one not infrequently
hears language which implies that this is the
only sphere in which knowledge can attain really valuable
\marginpar{418}
results. But to refute this rash assertion the sciences of
history, ethics, sociology present themselves, with the protest
that the character of science cannot be denied to these
disciplines, except on the principle that among the objects of
experience only those can be truly known which are \textit{unlike}
us in their inner nature, as being impersonal or inanimate;
while personality, or mind itself, is the one unknowable
thing in the world. The proverb that like is known by
like is a safer guide than any theory which thus threatens
to make cognition stand on its head. Accordingly, when
theology professes to apprehend realities of a personal, and
therefore of an unseen and supersensible kind, we shall
not be daunted by the objection that no genuine science
can travel beyond the categories of time and space.

A full and satisfactory treatment of this subject, it is
true, can be given, or at least attempted, only at a later
point, when we face the problems, as numerous as they are
difficult, which cluster round the nature of religious knowledge.
But even here it may be said that science, in the
only sense in which it is worth while to use the word, is
simply the persistent effort to reach an orderly interpretation
of experience, the effort of the mind to discover, in the
course and movement of all outward things, intellectual
principles which are identical with its own. The experience
under review may be sensible, or social, or ethical, or intellectual,
or religious; but in each case what happens is that
a science or a group of sciences applies itself persistently to
reduce the facts to intelligibility by the formation of hypotheses
or theories, and the unceasing alteration and correction
of these theories, till they correspond with and account
for the experiential facts from which they set out. Take
away the experience, that is, and you take away the science;
for you quench the only interest which the mind can possibly
feel in the scientific process---the interest, namely, of
explaining facts which have actually entered into our life.
\marginpar{419}
These facts, as I have hinted, may be placed in a graduated
scale of value and reality, according as they concern merely
some isolated intellectual faculty, or appeal to our entire
personality. Mathematics, the most abstract of sciences,
is an instance of the one class; ethics may be taken as
illustrative of the other. And what one is moved to protest
against is the tendency to restrict the term science to
the most abstract and hypothetical conceptions of the
mind, and to refuse it to those which are growingly concrete,
growingly in contact with reality. There never has
been a perfect line in real nature, a line, that is, which is all
that a line should be; and our reasonings about lines, therefore,
if applied to the actual world of fact, require instantly to
be modified and qualified in many ways; otherwise, as we all
know, they would issue in error and absurdity.\footnote{
Cf. Illingworth, \textit{Reason and Revelation}.
} 
On the
other hand, there has once been a perfect human Life, a life
which was all that a life should ever be; so far from our
thoughts about it being too ideal for the actuality, we
know that nothing we can ever think exhausts or even
adumbrates the fulness that was in Him. And if science
means concrete knowledge,---knowledge, valid and certain,
of things as they actually exist,---what justice is there in
calling trigonometry science and refusing the name to
Christian theology? I mean, is it possible to deny that the
experience in the one case is infinitely more real and concrete
than in the other; and that the richer species of
cognition has the better claim to rank as knowledge
proper?

Still, while this is true, we need not fall into the error
of the intellectualist, or be confused by a plausible and
therefore most malign fallacy which gave more trouble, perhaps,
to a former generation than it appears to do to ours.
For it used to be affirmed, especially by writers of the
Hegelian school, that the task of Dogmatic is to raise faith
\marginpar{420} 
to the level and the insight of knowledge. The formula
has its uses, but it is at least liable to misinterpretation.
If it means that Dogmatic strives to cast the utterances of
na\"{i}ve piety into intellectual form, that is true enough, as
it is also obvious enough. For example, it is often needful
to strip off the dress of figure and imagery worn by
religious ideas in popular usage, before they can be fitted
into their place in a theological system; and if it only be
kept in mind that the figurative character of certain
religious ideas is really their salvation, and gives them
their hold upon our hearts, no harm will come of the
application of this principle. But in the hands of most of
its champions, the principle meant something very different.
It meant that the specifically religious element in
belief was to be evaporated into metaphysic. Now,
without losing our way in the details of criticism, we may
at least say that this attempt to turn the theologian into a
pure metaphysician offends against the fundamental maxim
that the student of Dogmatic is no dispassionate scientist,
but a servant of the Church of Christ. He is a believer;
the faith once delivered to the saints is his faith. For him,
as for the Apostles, personal union to the living Christ is
not merely the secret of the Christian life; it is also the
organizing principle of Christian thought and theory. And
thus the propositions of a true Dogmatic still remain the
utterances of personal faith as really as the appeal of the
evangelist, or the prayers of the simplest believer in his
cottage among the hills. Indeed it would not be too much
to say that the doctrine which cannot be turned into a
sermon, and preached, is not worth its place in a system of
Dogmatic.

The relations of theology and philosophy, however, are
not, I need hardly say, of a purely negative or exclusive
sort. The practice of most theologians of repute, when
embarking on their enterprises in divinity, has been to
\marginpar{421}
justify the existence of systematic theology by an appeal to
considerations of a more or less philosophical kind; and
Ritschl, while honourably known for his services in banishing
speculative rationalism from the domain of Christian
doctrine, was himself no exception to the rule. Every one
who begins to theologise feels how strong is the demand of
intelligence for rational unity, the inconsequence of any
abrupt cessation of the work of reflection, the necessity,
above all, of some criterion which will distinguish the true
elements of religious experience from the false. And these
are philosophical ideas. In dealing with its special object,
theology claims to possess no special organ of knowledge
by an appeal to which inconvenient questions may
be evaded. It works with the ordinary instruments of
thought. No doubt valuable results can be expected
only from those who sympathize with the aspirations of
faith; but the same may be said, \textit{mutatis mutandis}, of
{\ae}sthetics or sociology or ethics.

Furthermore, religious experience has a cognitive side.
The judgments of faith claim to be true of a reality, of a
system of things, existing quite independently of our interest
in it. And to conceive this world of divine and spiritual
being at all, we need conceptions which are philosophical
if they are anything. What other name can be given to
such ideas as personality, or end, or cause? It is open to
a theologian, indeed, to repudiate the meaning assigned to
terms like these by the dominant philosophical school, but
the modifications he may propose leave them as metaphysical
as ever. Both theology and philosophy, again, are bound
to discuss such questions as the possibility of miracle, or
the theoretical efficacy of proofs of the existence of God.
And while the argument in each case may take a different
route, there is no difference of kind between the principles
they apply, the criteria they seek to conform to, or the idea
of truth which obtains in each department. Christian
\marginpar{422}
theology has refused, and refused rightly, to submit to the
tyranny of any particular system of metaphysics, or to use
no terms but those that might be licensed by the philosophy
of the day. Yet it has done so from no aversion to the
general method of philosophy, which it accepts as its own,
but from the conviction that the system in question has
done violence to certain elements in faith by forcing them
into logical formulas too narrow for their content.

Again, it would be ungrateful to forget that the long
labour of philosophy has done a great, or rather an inestimable,
service to theology by clarifying and elaborating
a more or less complete set of ideas and technical terms
which enable the modern divine to do his work. Putting
eccentricities aside, it may be said that we build upon the
assured results of logic, psychology, and ethics. And in this
region, we do well to keep gratefully in mind the intellectual
toil of the Middle Ages, when so much was done to survey
the continent of mind, and to estimate its logical potentialities.
No doubt the schoolmen had their limitations;
their Platonism on the one hand, and their Scepticism on
the other, made it all but impossible that they should do
justice to the new and revolutionary truth of Christianity.
But within these limits their work was of noble proportions,
and it is a writer of real insight who has said that ``in
raising their theologico-philosophical structures they were
fellow-workers with the architects of the great Gothic
cathedrals and monastic churches of that very age. And
though modern thought passed into fresh fields by rejecting
considerable masses of their work, yet in certain main
issues the rejections were much less extensive than is
commonly supposed, and many of their leading thoughts
persisted under new guises, and persist still.''\footnote{
Caldecott, \textit{Selections from the Literature of Theism}, p. 38.
} 
A good
deal of specious nonsense, indeed, has been talked about
the dry and futile discussions of Scholasticism; although I
\marginpar{423}
observe that this is not the language held by those who
have gone most deeply into the subject.

On the other hand, however, this immense difference
remains, and will ever remain, to mark theology off from
philosophy, that theology is not so much concerned to
discover truth, as to interpret it. For the theologian
starts from a great datum. On the objective side he
starts from the Gospel as realized and embodied in the
historical Person of Jesus Christ, the Crucified and Risen
Lord; on the subjective side, be starts from the consciousness
of redemption through union to Christ. This
is the situation which he is brought in to explain; and the
Christian mind has no use for any theology that does not
accept and deal with these facts as it finds them, or that
seeks to persuade the simple believer that in giving Jesus
a place, and a central place, in the Gospel, be is only the
victim of decadent Greek metaphysics. Christ, and the
absolute certainty of saving union to Christ, constitute our
immovable point of departure; and thus it is not surprising
that speculative systems, even though to some extent
they employ the same principles of thought and criticism,
should occasionally arrive at results so unrecognizable, so
unlike the Christian verities as we find them in the writings
of St. Paul and St. John. For they are really building with
quite other materials than the Christian thinker, and on quite
other foundations. Theology, I mean, when properly aware
that its business is to deal with the specifically Christian
experience, takes the unconditional truth and value of the
revelation in Jesus for granted; whereas for pure philosophy
this is still an open question. \textit{Philosophia}, as the old
maxim has it, \textit{philosophia veritatem quaerit, religio sc. religio
revelata veritatem possidet}. This is frequently demurred
to as an overstatement, and even cited as a typical instance
of how superciliously self-assertive theology can be. But
obviously there is no choice; you cannot believe in Christianity 
\marginpar{424}
at all without believing that it is the truth which is
at the root of everything. Moreover, there are words of
Jesus Himself on record which make any other view a
treachery to the faith. We cannot forget that He said:
``Neither knoweth any man the Father save the Son, and
he to whomsoever the Son will reveal Him,'' or again, ``I
am the way, and the truth, and the life.'' \textit{There} is the note
of absolute and irrefragable certitude; and theology is false
to its own duty and honour if it fails to preserve that note,
not indeed as pertaining to its theoretical constructions, but
as an inherent quality of the basis of fact on which it stands.
Nor is anything more sure than the fact that you cannot
meet the perplexities of men who are baffled by the enigmas
of all this unintelligible world, except by holding forth to
them a Gospel which is not only very great and very wonderful,
but indisputably true. A conjecture may have its
own charm as an intellectual toy. The pastime of forming
religious hypotheses, and dissolving them again, is one of
the most fascinating in which the dialectical voluptuary can
engage. But moments come in every life when their essential
hollowness is felt, and felt with a certain shame. In
hours of fierce temptation a theory which is no more than
a theory is but a broken reed, which will pierce and poison
the hand that leans on it. And still more impotent do we
feel hypotheses to be when we are called in to aid the man
whose faith is being assailed by doubt. You must have
some sure word to offer him; you cannot press a conjecture;
for, in the words of Professor James, ``who says hypothesis
renounces the ambition to be coercive in his arguments.''
And another brilliant and suggestive writer has touched the
same point, and named it the problem of our time. ``There
is abroad among thinking men of all schools,'' he says,
``a greater consciousness of the mystery of existence.
There is also an increased anxiety for some means by which
to come to terms with that mystery. If Christianity is to
\marginpar{425}
win and hold the allegiance of the modern mind, it must be
able, if not to solve the great problems, at least to make
them endurable.''\footnote{
C. F. D'Arcy, \textit{Idealism and Theology}, p. 168.
} 
Endurable they can be made only by
the gift of a great all-embracing assurance, and this it is
the task of Christian doctrine to bestow. Let it consent to
lay aside the note of certainty, and the reason for its very
being is gone.

I have tried to urge that the distinction between Christian
doctrine and philosophy is at bottom, at least very largely,
the distinction between certainty and conjecture. But of
course this dictum would have to be largely qualified. And
perhaps the easiest way in which to suggest the proper
qualification is to go on to say that the same distinction
must be re-introduced within the sphere of theology itself.
Here, too, we must clearly distinguish the central orb of
light from the penumbral haze by which it is surrounded; or,
as it has been put elsewhere, ``we must map off the realm
of certitudes from the region in which assurance is unattainable,
and in which variety of speculation is admissible.''
What I mean may become clearer if I take an example, and
the example I will take is the doctrine of the Atonement.
We are told by many voices, and in particular by one voice
of singular clearness and power, that in regard to this topic
it is really illegitimate to distinguish the fact of the Atonement
from the theory. ``There is no such thing conceivable,'' 
it is said, ``as a fact of which there is no theory, or
even a fact of which \textit{we} have no theory; such a thing
could never enter \textit{our} world at all; if there could be such a
thing, it would be so far from having the virtue in it to
redeem us from sin, that it would have no interest for us
and no effect upon us at all.''\footnote{
Denny, \textit{Studies in Theology}, p. 106.
}
In a sense this is very true;
only, if the practice of human life is any guide, it is not so
true as its opposite. ``In every other province of human
\marginpar{426}
thought,'' says Dr. Dale, ``we ascertain the facts first---make
sure of \textit{them}---and try to explain them afterwards.
We never deny the facts because we find them inexplicable.
. . . And it may be that we shall find ourselves unable to
give any account of the relation between the death of
Christ and the forgiveness of sin; and yet the fact that the
death of Christ is the ground of forgiveness may be so
certain to us as to be a great power in life.''\footnote{
\textit{Christian Doctrine}, p. 223.
}
It is true
that the mind finds it hard to rest satisfied with the fact.
It is true that it demands a doctrine, an explanation, a
complete theory. But then the mind demands many things,
in this life of guilt and clouded vision, that it simply cannot
have. It may have adumbrations of a theory; it does have
them; only we may be sure in advance that the great
reality has, depths in it which our line is too short to
fathom. And while holding, as I do, that ``Christ bore our
sins in His own body on the tree,'' that we have redemption
through His blood, the forgiveness of sins, and that the doctrine
which denies this is not recognizable as New Testament
Christianity, I still find in the believing consciousness something
which echoes to the declaration that ``all that has ever
been written on the subject only leaves behind the sense of
the wonder of the mystery, and every explanation that has
been attempted is overthrown with an ease which warns us
that explanation is impossible. Every statement of the
doctrine which has ever yet been made always contains
those self-contradictions, those manifest breaches of the
plainest rules of logic, which indicate that the human
intellect is baffled.'' This also is an overstatement of the
case; it is not possible that the meaning of the Cross
should be wrapped in pure impenetrable darkness; we have
the elements of a doctrine, and something more; yet it is a
side of the truth which we must vindicate over and over
again. The affectation of a spurious certainty regarding
\marginpar{427}
what after all are no more than intellectual hypotheses, it is
probable, has had too much to do with the aversion of the
general mind from systematic theology. And the refusal to
bind the fact of the Atonement indivisibly to all the details
and all the refinements of any theory is the first step in
assuring the real progress of the theory itself, as it freely
strives ever more adequately to interpret the infinite fact.
While at the same time it escapes the real, and sometimes
the terrible, danger of leading men to believe that when
their intellectual conceptions of the Atonement fall in
ruins, they forfeit thereby the benefits that are ours
through the Cross, or have lost the right to believe on the
Lamb of God, that taketh away the sin of the world.
This may be enough to indicate the need for drawing the
distinction between certainty and theoretical construction,
even within the precincts of theology itself.

Passing then from the relations of theology and philosophy,
let us glance, ere we close, at a point of somewhat kindred
interest. If Dogmatic is not a philosophical, is it then
a historical science? Now we were led to note that a
real difficulty emerges when it is asked how a science can
deal with realities of an unseen and supersensible kind,
and it is under the pressure of a similar difficulty, no doubt,
that some have been moved to define Dogmatic as a purely
historical discipline. Thus, for example, it has been urged,
as by Schleiermacher and in a modified fashion by Rothe,
that the task of Dogmatic is to give an orderly and articulate
view of the doctrines prevailing in a specific Church
at a specific time. But this attempt to place our science
under the general heading of history has had little success;
and for two reasons. In the first place, it has become increasingly
clear that theology---whose object is not Church
doctrine---but Divine revelation, is dealing with realities
which, although they entered the stream of human life at a
particular spot in the past, and consequently are historical,
\marginpar{428}
yet arise above the limits of mere history, and belong to all
time and all existence. Jesus Christ is indeed a figure in
the annals of the world; His name is found upon the pages
of ancient authors; yet it is the experience of countless
multitudes to-day that He is the most urgent and substantial
reality of their inward life. Mere history has no rules
for dealing with such a phenomenon; and the historian who
understands the limits of his province is quite aware that
it is \textit{ultra vires} for him to estimate aright a Person who
is thus a historical datum, and yet claims to be of infinite
significance for every soul that has ever lived. And in the
second place, Dogmatic refuses to be classed among the
sciences of history, because it cherishes ideals. It is interested
not merely in what has been believed, but even
more in what ought to be believed. It is a normative
science; it sets up a standard of truth and value. It
criticises the past. That criticism must be full of sympathy,
or it will do no good; it must be full of humility, or it will
do incalculable harm; but these conditions, difficult as
they are, still may be fulfilled. It is another question from
what source the norm of Christian doctrine should be
drawn. In point of fact, of course, it has been drawn from
a variety of sources---from Scripture, as a presumably consistent
whole; from some selected portion of Scripture,
which has been assigned decisive importance; or from the
contents or the presuppositions of an ideal Christian experience.
But whatever our conclusions on this thorny
problem, at least the impossibility of ranking a normative
science as historical is transparently clear.

This really implies, I need hardly say, that Dogmatic, as
a science which is working towards an ideal, is bound to
contain an element which is so far subjective and mutable.
For naturally each theologian will put in operation a different
set of criteria. He cannot think with any other
mind than his own; he cannot live in any other age than
\marginpar{429}
his own; he cannot change experiences with any one else
not even St. Augustine; and it follows that his attitude to
the traditions of the past must be a personal one. His
use of Scripture, for example, will of necessity be modified
by the position and progress of Biblical science in his day.
Whereas a writer belonging to the third century would use
Scripture, by a kind of second nature, in a predominantly
allegorical sense, the historical and scientific methods of
modern exegesis have made this once for all impossible.
And if it be said that this appears to commit the theology
of the Church to the vagaries of mere caprice, and the cry
be raised for some inflexible rule by which to measure the
correctness of opinions, it must be replied that no \textit{legal}
guarantee for unchanging orthodoxy can ever be given.
Nothing in Christianity, let us be thankful, can be guaranteed
in that way. At all events, if you call in the law, in
whatever form, to protect the Gospel, you have to pay
heavily for it in the end.\footnote{
Cf. Denny, \textit{Studies in Theology}, p. 195.
} 
There are better sureties, too,
within the reach of the Christian mind. We have the
promise of the Holy Spirit, to lead the Church into all truth;
we have the Word of God, which liveth and abideth for
ever, and to which the Spirit bears witness perpetually in
the hearts of men. These are the real,---these, when we speak
strictly, are the only and the sufficient---guarantees that the
mind of the believer, working freely on its data, will reach
conclusions that are in line with the great faith of the
past.

But in accepting this, which is after all only one of the
honourable risks of Protestantism, we are putting our trust,
not in the letter of symbol or confession, but in the life
and power of the Holy Ghost. And as we look back, over
the chequered history of doctrinal development, we seem to
mark His divine guidance as it leads theologians, gradually,
and doubtless with many times of retrogression, to be
\marginpar{430}
resolute and thorough in the effort to look at every doctrine
in the pure light of the Person of Christ. ``He shall take
of mine, and shall show it unto you''---the promise is being
fulfilled unto this day. And so far as it is fulfilled in our
experience, as believers and as students of theology, it will
bring us to apply the principle, unflinchingly but not, I
trust, hastily, or without sympathetic care, that no doctrine
can retain its place in the Christian creed save those
which strike their roots deep down into the living union
that binds the Christian to his Lord. It must be
left, however, to the believing instinct of the individual to
say when this condition is satisfied. And thus once more
we turn back to the truth that the theologian must be a
Christian, in frank and warm accord with the Church's
common faith. The notion, widely spread though it be
in Scotland, that any given man is equally fit to form a
judgment on doctrine with any other, is a pure mistake,
though it is one of which we find it very hard to clear our
minds. There are those who have no right to an opinion
respecting Christian truth; they have never sought or
gained an experimental knowledge of Christ's redeeming
grace; and we know that the secret of the Lord is with
them that fear Him. But to the man who understands
what he is doing, his theology is part of his Christian life.
As he realizes afresh every day what God has done for him
in Jesus, he feels that he has within his grasp the one
standard of all value and all reality; and that without the
decisive guidance afforded by this inward certainty, men
are only playing at theology. Yes! the knowledge we
have of divine truth will to the end be relative and in part;
but the conviction with which we hold it may still be in
essence absolute.

And it is thus, after all, that theology serves the Church---by
feeding and illumining the new conviction that fills
the Christian mind. It is thus, I repeat, that it serves
\marginpar{431}
the Church; for conviction is the true spring and cause of
preaching; nothing else will turn a doctrine into a Gospel.
If Christianity is true, then it is designed for proclamation;
it has not begun to be what it aims at being until it is
proclaimed. And for this reason the science we study here is
alive and wholesome only as it springs from an indestructible
certainty that in Jesus Christ we have God personally
present in the world for the rescue and salvation of men,
and as it moves men in consequence to go out to their
fellows, making great affirmations as to the grace that is
in Him for a world of sin. It is my hope and prayer that
in the Dogmatic class-room still, as throughout the past,
an impression of Christianity may be given which will
make men eager to preach it. There we shall be occupied,
not with the puzzles and enigmas of human thought,
which too often reveal to us our weakness, but with the
glorious Gospel of the blessed God, which is a revelation
of our strength; for strong we are indeed if God has
love and we have faith. And while we shall never, I
trust, forget the limitations of our insight, yet we shall
take for granted from the first that God has made clear
and simple what He meant by Christ, and that He meant
salvation. We shall build upon the belief that it was the
need and darkness of man that bespoke the compassion
of the Most High, and that what He has given so freely in
the Person of His Son is in the main not a problem to be
wrestled with, but a gift to be received. For to treat these
matters as open questions would be to affect ignorance of
what every simple Christian knows perfectly well. It is a
more excellent way, surely, to assume the Christian faith
as the final truth for man, and diligently to search out
its implications.

\begin{flushright}
\textsc{H. R. Mackintosh}
\end{flushright}


\end{document}